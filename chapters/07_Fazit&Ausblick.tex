\chapter{Fazit und Ausblick} \label{chap:fazit}

\section{Zusammenfassung der Ergebnisse} Die vorliegende Arbeit beschäftigte sich mit der Entwicklung eines Systems zur Visualisierung von Mitarbeitendengesprächsdaten. Das Ziel bestand darin, datenbasierte Entscheidungsprozesse im HR-Kontext zu unterstützen und durch innovative Technologien effizientere Lösungen zu schaffen. Die Arbeit demonstrierte, wie durch den Einsatz moderner Technologien wie React, .NET Core und Microsoft Azure eine flexible und skalierbare Grundlage geschaffen werden konnte. Die Integration von Visualisierungstools, insbesondere Donutcharts und Radarcharts, erwies sich als besonders hilfreich, um komplexe Daten anschaulich und verständlich darzustellen. Diese Ansätze ermöglichen eine vereinfachte Interpretation der Daten und fördern eine gezielte Ableitung strategischer Maßnahmen.

Besonderer Fokus lag auf der Benutzerfreundlichkeit und Sicherheit des Systems. Durch die Kombination von Echtzeitdatenverarbeitung, interaktiven Funktionen und einer intuitiven Benutzeroberfläche konnten sowohl technische als auch praktische Anforderungen erfüllt werden. Die Implementierung eines relationalen Datenbanksystems sorgte für eine strukturierte und effiziente Speicherung der Gesprächsdaten.

Die theoretischen Grundlagen und technologischen Analysen belegten zudem, dass datenbasierte Ansätze nicht nur die Darstellung von Informationen, sondern auch die Identifikation von Kompetenzprofilen, die Verfolgung individueller Entwicklungsziele und die Abstimmung mit strategischen Unternehmenszielen erheblich verbessern können. Somit stellt das entwickelte System einen relevanten Beitrag zur Digitalisierung von HR-Prozessen dar.

\section{Potenzielle Weiterentwicklungen} Die Arbeit hat gezeigt, dass datenbasierte Visualisierungssysteme im HR-Bereich ein großes Potenzial bieten. Dennoch gibt es zahlreiche Möglichkeiten, das entwickelte System weiterzuentwickeln: \begin{itemize} \item \textbf{Integration von Machine-Learning-Modellen:} Der Einsatz prädiktiver Algorithmen könnte dabei helfen, Muster in den Daten zu erkennen und zukünftige Entwicklungen, wie die Wirkung bestimmter Maßnahmen, vorherzusagen. \item \textbf{Erweiterung auf andere HR-Prozesse:} Neben Mitarbeitendengesprächen könnten auch Prozesse wie Onboarding, Schulungsplanung oder Teamanalysen von einer datenbasierten Lösung profitieren. \item \textbf{Internationalisierung:} Die Mehrsprachigkeit des Systems wäre eine wichtige Erweiterung, um das Tool in global tätigen Unternehmen einzusetzen. \item \textbf{Integration in bestehende HR-Software:} Eine tiefere Verknüpfung mit Tools wie HRworks oder Evalea würde die Akzeptanz und Effizienz des Systems erhöhen. \item \textbf{Erweiterte Visualisierungsoptionen:} Die Implementierung zusätzlicher Diagrammtypen, wie beispielsweise Bubble Charts oder Heatmaps, könnte die Analyseflexibilität weiter steigern. \item \textbf{Feldtests in Unternehmen:} Um die Praxistauglichkeit des Systems zu validieren, könnten umfangreiche Tests in realen Anwendungsszenarien durchgeführt werden. \end{itemize}

\section{Persönlicher Ausblick} Die Entwicklung dieses Systems bot nicht nur einen tiefen Einblick in moderne Technologien und deren Anwendung im HR-Bereich, sondern auch die Möglichkeit, praxisnahe Lösungen für komplexe Probleme zu erarbeiten. Besonders die Verbindung von theoretischem Wissen mit praktischen Umsetzungen stellte einen zentralen Lerngewinn dar. Die gewonnenen Erkenntnisse und Erfahrungen werden eine wertvolle Grundlage für zukünftige Projekte und berufliche Herausforderungen sein.

Die Arbeit unterstreicht die Relevanz interdisziplinärer Ansätze, bei denen technologische Innovationen auf die Bedürfnisse der Arbeitswelt abgestimmt werden. Moderne HR-Management-Systeme können nicht nur die Effizienz von Prozessen steigern, sondern auch die Qualität von Entscheidungen verbessern und eine transparente Kommunikation zwischen Führungskräften und Mitarbeitenden fördern. Als langfristiges Ziel sehe ich die Weiterentwicklung datengetriebener Systeme, die Unternehmen dabei unterstützen, ihre strategischen Ziele zu erreichen und gleichzeitig die Zufriedenheit der Mitarbeitenden zu erhöhen.

Abschließend bleibt festzuhalten, dass die digitale Transformation im HR-Bereich große Chancen bietet. Systeme wie das in dieser Arbeit entwickelte Visualisierungstool können nicht nur bestehende Prozesse optimieren, sondern auch als Basis für weitere Innovationen dienen. Damit leistet die Arbeit einen wichtigen Beitrag zur Weiterentwicklung moderner HR-Strategien und bietet zugleich eine fundierte Grundlage für zukünftige Forschung und Praxis.