\chapter{Konzeption des Systems}
\label{chap:konzeption}

\section{Systemarchitektur}
\subsection{Architekturüberblick}
\begin{itemize}
    \item \textbf{Frontend:} Benutzerinteraktionen und Visualisierung mit React.
    \item \textbf{Backend:} Datenverarbeitung und -verwaltung mit .NET Core.
    \item \textbf{Datenbank:} Speicherung der Daten in Azure SQL.
    \item \textbf{Datenfluss:} Kommunikation zwischen Frontend und Backend über RESTful APIs.
\end{itemize}

\section{Datenmodell}
\begin{itemize}
    \item \textbf{Mitarbeiterdaten:} ID, Name, Position.
    \item \textbf{Gesprächsdaten:} Mitarbeiter-ID, Datum, Bewertungspunkte.
    \item \textbf{Visualisierungsparameter:} Diagrammtypen, Datenquellen.
\end{itemize}
Beziehungen werden mithilfe eines UML-Diagramms dargestellt.

\section{Visualisierungsdesign}
\subsection{Diagrammtypen}
\begin{itemize}
    \item \textbf{Donut-Charts:} Überblick über den Status von Zielvereinbarungen.
    \item \textbf{Radarcharts:} Vergleich der Leistung in verschiedenen Kategorien.
\end{itemize}

\subsection{UI/UX-Überlegungen}
\begin{itemize}
    \item Klar strukturierte Benutzeroberfläche.
    \item Fokus auf einfache Bedienung und schnelle Navigation.
\end{itemize}

\section{Sicherheitskonzepte}
\begin{itemize}
    \item \textbf{Datenverschlüsselung:} SSL für Datenübertragung, AES-256 für Datenspeicherung.
    \item \textbf{Benutzerrechte:} Zugriffskontrolle basierend auf Rollen (z. B. Admin, Benutzer).
    \item \textbf{Authentifizierung:} OAuth 2.0 für sichere Anmeldung.
\end{itemize}