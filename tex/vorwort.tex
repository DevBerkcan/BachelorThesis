\chapter*{Vorwort}
\addcontentsline{toc}{chapter}{Vorwort} % Fügt das Vorwort ins Inhaltsverzeichnis ein

 Die Bachelorarbeit wurde während meines Informatikstudium (B.Sc.) an der Hochschule Bochum erstellt und befasst sich mit dem Thema “Analyse und Entwicklung eines Systems zur Visualisierung von Mitarbeitendengesprächsd at en". Dieses Thema wurde gewählt aufgrund seiner spannenden und praxisnah en Forschungsbedeutung sowie der direkten Verbindung zu einer von mir entwickelten Anwendung. Diese Anwendung ist aus einem konkreten praktischen Bedarf entstanden und wurde im Verlauf der Arbeit kontinui erlich verbessert, um den wissenschaftlichen und praktischen Anforderungen gerecht zu werden. 
 
 Die Schaffung dieses Systems ermöglichte nicht nur einen umfassenden Einblick und Verständnis moderner Technologien und ihrer Anwendung im HR-Bereich, sondern biete auch die Gelegenheit zur Entwicklung praxisnaher Lösungen für anspruchsvolle Herausforderungen. Insbesondere die Verknüpfung von theoretischem Wissen mit praktischer Anwendung erwies sich als wertvolle Lernerfahrung. 
Die Studie betont die Wichtigkeit eines interdisziplinären Ansatzes, bei dem technologische Fortschritte an die Anforderungen der Arbeitsumgebung angepasst werden. Moderne HR-Managementsystem können nicht nur Prozesse effektiver gestalten, sondern auch die Qualität von Entscheidungen erhöhen und eine klare Kommunikation zwischen Führungskräften und Mitarbeitern unterstützen.  

Zum Abschluss zeigt diese Studie auf eindrucksvolle Weise auf, wie moderne Technologien und datengestützte Visualisierungen das Konzept der Mitarbeitergespräche nachhaltig verändern können. Die gewonnen Erkenntisse und praktischen Erfahrungen bieten eine solide Basis für die Weiterentwicklung dieser Ideale innerhalb zukünftiger Projekte und beruflicher Tätigkeitsfeldern. Auf diese Weise leistet die Studie einen relevantien Beitrag zur Progression modernster HR-Konzepte und liefert gleichzeitig eine fundierte Grundlage für künftige Forschung und Praxisanwendungen. 

Mein besonderer Dank geht an Frau Anja Tenberge als meine Erstprüferin für ihre konstruktive Unterstützung und hilfreichen Anregungen während meiner Arbeitserstellung.Wertschätzen möchte ich ebenfalls Herrn Sascha Bajonczak, meinem Zweitprüfer, sowie alle Kolleginnen und Kollegen für ihre Unterstützung während des Entstehungsprozesses dieser Arbeit.Darüber hinaus möchte ich meiner Familie und meinen Freunden für ihre fortwährende Unterstützung dankend erwähnen,die mir in dieser intensiven Zeit stets zur Seite stand(en).

\vspace{1cm}

\begin{flushright}
Düsseldorf, Januar 2025 \\
Berk-Can Atesoglu
\end{flushright}
