\chapter{Einleitung}
	\label{chap:einleitung}
	
	Das Wort Physik bedeutet in etwa \enquote{Naturlehre} und bezeichnete ursprünglich eine Disziplin der Philosophie. Es ist überliefert, dass bereits die Naturphilosophen der Antike über die Natur nachdachten, sie beschrieben, erklärten und deuteten -- allerdings in Form von Texten. Erst die Naturforscher des 16.\,und 17.\,Jahrhunderts entdeckten, dass sich die Mathematik hervorragend eignet, um ihre Hypothesen und Schlussfolgerungen präzise zu formulieren. Diese Entdeckung gilt als Geburtsstunde der Physik als Naturwissenschaft. Da Physik heute nicht mehr auf qualitativen Beschreibungen (\zB \enquote{Er läuft langsam.}), sondern auf quantitativen Aussagen (\zB \enquote{Er läuft fünf Kilometer pro Stunde.}) beruht, wird sie häufig als \emph{exakte} Wissenschaft bezeichnet. %
	
	\section{Fließtext}
		\label{sec:fliesstext}
		Den Fließtext können Sie einfach runterschreiben. Eine oder mehr Leerzeilen im Code bewirken einen Absatz im Text. Beim Schreiben eines längeren Textes ist es oft hilfreich, wenn man sich ToDos \todo{Das ist ein ToDo.} notieren kann. %
		
		Wenn Sie Kapitel, Abschnitte usw. mit Labeln versehen, können Sie auf die jeweiligen Nummern referenzieren. Wir befinden uns \zB in \autoref{chap:einleitung} und \autoref{sec:fliesstext}. %
		
		Natürlich gibt es auch Hilfsmittel für Aufzählungen: %
		\begin{itemize}
			\item Das \dots
			\item ist \dots
			\item eine \dots
			\item Aufzählung.
		\end{itemize}
			
	\section{Formeln und Einheiten}
		Formeln und Einheiten können ganz besonders schön aussehen. \textcolor{gray}{\lipsum[1]} %
				
		\subsection{Formeln}
			Formeln werden automatisch nummeriert:
			\begin{equation}
				\label{eq:pythagoras}
				a^2 + b^2 = c^2 \,.
			\end{equation}
			Auch die Formelnummern können Sie referenzieren: \eqref{eq:pythagoras}.
			
			Eine hilfreiche Übersicht über die Möglichkeiten des Formelsatzes finden Sie unter \url{https://en.wikibooks.org/wiki/LaTeX/Mathematics}. %
		
		
		\subsection{Einheiten}
			Wenn Sie das \texttt{siunitx}-Paket benutzen, können Sie Einheiten sowohl im Fließtext als auch innerhalb von Formelumgebungen nutzen. Hier etwa im Fließtext: \SI{5}{\m\per\s}. Oder in der Formelumgebung: %
			\begin{equation}
				\SI{2000000}{\J} = \SI{2e6}{\J} = \SI{2}{\mega\J} \,. %
			\end{equation}
			
	\section{Abbildungen und Tabellen}
	
		\textcolor{gray}{\lipsum[1]}
	
		\begin{figure}
			\centering
			\includegraphics[width=0.2\linewidth]{Testbild.jpg}
			\caption{Das ist das Logo der Hochschule Bochum.}
			\label{fig:logo}
		\end{figure}
		
		Abbildungen und Tabellen werden in \LaTeX automatisch platziert. Grundsätzlich sollte man bis ganz zum Schluss warten, bis man beginnt, die automatische Platzierung zu manipulieren. Bei \autoref{fig:logo} handelt es sich um eine normale Abbildung mit einer Bildunterschrift. %
		
		\textcolor{gray}{\lipsum[1-4]}
		
		Bei recht schmalen Bildern sieht es oft schöner aus, wenn der Text neben dem Bild steht (wie etwa bei \autoref{fig:logo2}). %
		
		\begin{SCfigure}[3]
			\includegraphics[width=0.2\linewidth]{Testbild.jpg}
			\caption{Das ist ebenfalls das Logo der Hochschule Bochum. Hier steht der Text jedoch neben dem Bild.}%
			\label{fig:logo2}
		\end{SCfigure}
		
		\textcolor{gray}{\lipsum[1-4]}
		
		Tabellen sollten grundsätzlich keine senkrechten Striche und möglichst wenig waagerechte Striche haben. Ganz gut gelingt das mit dem \texttt{booktabs}-Paket. Ein Beispiel finden Sie in \autoref{tab:einkommen}. Wenn Sie Tabellen erstellen möchten, die automatisch eine bestimmte Breite einnehmen, hilft das Paket \texttt{tabularx}. %
		
		\begin{table}
			\centering
			\begin{tabular}{lrr} 
				\toprule
				\multicolumn{2}{c}{Studium}\\ \cmidrule{1-2}
				Fach & Dauer & Einkommen (\euro{})\\ 
				\midrule 
				Info & 2 & 12,75 \\ \addlinespace
				MST & 6 & 8,20 \\ \addlinespace
				VWL & 14 & 10,00\\ 
				\bottomrule
			\end{tabular}
			\caption{Ein Beispiel für eine Tabelle.}%
			\label{tab:einkommen}
		\end{table}
		
		\textcolor{gray}{\lipsum[1]}
		
	\section{Literaturverweise}
		Wenn Sie Literaturverweise in einer \texttt{.bib}-Datei ablegen, können Sie auf diese referenzieren \cite{foo1999} und automatisch ein Literaturverzeichnis erstellen lassen. Diese Funktionalität kommt vom Paket \texttt{biblatex}. %
			