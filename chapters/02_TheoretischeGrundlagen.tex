\chapter{Theoretische Grundlagen}
\label{chap:theoretische-grundlagen}

Die individuelle Entwicklung von Mitarbeitenden erfolgt maßgeblich durch zielgerichtete Gespräche und datengestützte Feedbackprozesse. Die folgenden Abschnitte beleuchten die Bedeutung von Mitarbeitendengesprächen, die Rolle der Datenvisualisierung in diesem Kontext sowie die technologische Basis für deren Implementierung. Hierbei wird auf die Verknüpfung von strategischen Unternehmenszielen und persönlichen Entwicklungsmöglichkeiten eingegangen, um eine nachhaltige Verbesserung der Teamdynamik und Leistungseffizienz zu fördern. Diese theoretischen Grundlagen bilden das Fundament für die anschließende Analyse und Entwicklung eines optimalen Systems zur Visualisierung von Mitarbeitendengesprächsdaten.

\section{Mitarbeitendengespräche und ihre Bedeutung}
Mitarbeitendengespräche, auch als „MAG“ bezeichnet, umfassen alle Gespräche, die Vorgesetzte mit Mitarbeitenden zu spezifischen Anlässen führen. Diese Gespräche dienen dazu:
\begin{itemize}
    \item Informationen auszutauschen,
    \item Wertschätzung, Lob oder Kritik zu äußern und
    \item Entwicklungsziele zu formulieren \cite{schober2008}.
\end{itemize}

Ein strukturierter Ablauf solcher Gespräche fördert die innerbetriebliche Kommunikation und die Bindung der Mitarbeitenden an das Unternehmen. Typische Elemente von Mitarbeitendengesprächen sind:
\begin{itemize}
    \item \textbf{Bewertung der bisherigen Zielerreichung:} Analyse des Fortschritts in Bezug auf zuvor gesetzte Ziele \cite{duarte2012performance}.
    \item \textbf{Setzen neuer Ziele:} Definition von SMART-Zielen (spezifisch, messbar, akzeptiert, realistisch, terminiert), die eine klare Richtung für die zukünftige Arbeit vorgeben \cite{duarte2012performance}.
    \item \textbf{Diskussion von Entwicklungsmaßnahmen:} Identifikation von Schulungsbedarf und Weiterbildungsangeboten zur Förderung von persönlichem und beruflichem Wachstum \cite{bryson2011employee}.
\end{itemize}

Eine strukturierte Gesprächsführung kann Verzerrungen durch subjektive Einschätzungen minimieren, insbesondere durch den Einsatz datenbasierter Methoden \cite{heikkila2018}. Dies unterstützt Führungskräfte dabei, fundierte Entscheidungen zu treffen, die sowohl die individuellen Ziele der Mitarbeitenden als auch die strategischen Unternehmensziele berücksichtigen \cite{barton2012}.

\section{Relevanz von Datenvisualisierung}
Datenvisualisierung ist ein zentraler Bestandteil moderner Analytik und spielt eine entscheidende Rolle bei der Interpretation komplexer Informationen. Sie unterstützt Führungskräfte dabei, datenbasierte Entscheidungen zu treffen, indem sie große Datenmengen in verständliche visuelle Darstellungen umwandelt \cite{kirk2016data}. 

\subsection*{Hauptvorteile der Datenvisualisierung}
\begin{itemize}
    \item \textbf{Verbesserte Entscheidungsfindung:} Daten werden schneller und effizienter interpretiert \cite{kirk2016data}.
    \item \textbf{Erkennung von Trends und Abweichungen:} Historische und aktuelle Daten können leicht verglichen werden, um Entwicklungen zu identifizieren \cite{ware2012information}.
    \item \textbf{Kommunikation und Präsentation:} Visualisierte Daten fördern eine klare Kommunikation zwischen Teams und Stakeholdern \cite{evergreen2016effective}.
\end{itemize}

Im Kontext von Mitarbeitendengesprächen sind Radarcharts und Donutcharts besonders geeignet:
\begin{itemize}
    \item \textbf{Radarcharts:} Vergleich mehrerer Leistungsdimensionen wie Teamarbeit, Effizienz und Pünktlichkeit \cite{heikkila2018}.
    \item \textbf{Donutcharts:} Übersicht über den Status von Zielvereinbarungen oder Fortschritten \cite{evergreen2016effective}.
\end{itemize}
\newpage

\section{Technologische Basis}
Die technologische Basis für ein System zur Visualisierung von Mitarbeitendengesprächsdaten ist entscheidend für dessen Funktionalität, Benutzerfreundlichkeit und Skalierbarkeit. Die ausgewählten Technologien umfassen:

\begin{itemize}
    \item \textbf{React:} Eine JavaScript-Bibliothek für die komponentenbasierte Entwicklung dynamischer Benutzeroberflächen \cite{stefanov2021react}.
    \item \textbf{TypeScript:} Erweiterung von JavaScript mit statischer Typisierung zur Verbesserung der Codequalität \cite{typeScriptDocumentation}.
    \item \textbf{.NET Core:} Plattformübergreifendes Framework für die Entwicklung skalierbarer und leistungsstarker Backend-Anwendungen \cite{microsoftDotNet}.
    \item \textbf{MSSQL:} Eine cloudbasierte Datenbanklösung, die Sicherheit und Skalierbarkeit für große Datenmengen bietet \cite{azureDocumentation}.
    \item \textbf{Azure Ressourcen:} Verschiedene Dienste von Microsoft Azure wurden in der Implementierung genutzt, um eine umfassende und skalierbare Lösung zu gewährleisten:
    \begin{itemize}
       \item \textbf{Entra ID (ehemals Azure Active Directory):} Entra ID bietet umfassende Authentifizierungs- und Autorisierungsdienste, einschließlich Multi-Faktor-Authentifizierung und rollenbasierter Zugriffskontrolle. Studien zeigen, dass die Verwendung solcher Systeme die Sicherheit und Effizienz in Unternehmen erheblich erhöht \cite{microsoftEntraID, woods2020authentication}.
\item \textbf{Microsoft Graph API:} Diese API ermöglicht die Interaktion mit verschiedenen Microsoft-Diensten, wie Benutzerdatenabfragen und Kalenderintegration. Laut aktuellen Forschungsergebnissen ist die Nutzung von Graph APIs ein Schlüssel für die Automatisierung und Integration in moderne Softwarelandschaften \cite{microsoftGraphAPI, smith2021graph}.
\item \textbf{Application Insights:} Ein Telemetriedienst, der zur Überwachung und Analyse der Anwendungsleistung eingesetzt wird. Studien belegen, dass Application Insights eine signifikante Verbesserung in der Fehlerdiagnose und Performanceanalyse moderner Cloud-Anwendungen ermöglicht \cite{microsoftAppInsights, li2021monitoring}.
\item \textbf{App Services:} Ein Telemetriedienst, der zur Überwachung und Analyse der Anwendungsleistung eingesetzt wird. Studien belegen, dass Application Insights eine signifikante Verbesserung in der Fehlerdiagnose und Performanceanalyse moderner Cloud-Anwendungen ermöglicht 
\item \textbf{Container Registry:} Ein Telemetriedienst, der zur Überwachung und Analyse der Anwendungsleistung eingesetzt wird. Studien belegen, dass Application Insights eine signifikante Verbesserung in der Fehlerdiagnose und Performanceanalyse moderner Cloud-Anwendungen ermöglicht 

\end{itemize}

\item \textbf{Git:} Verschiedene Dienste von Microsoft Azure wurden in der Implementierung genutzt, um eine umfassende und skalierbare Lösung zu gewährleisten
\item \textbf{DevOps:} Verschiedene Dienste von Microsoft Azure wurden in der Implementierung genutzt, um eine umfassende und skalierbare Lösung zu gewährleisten:
    \begin{itemize}    
    \item \textbf{CI/CD-Pipeline:} Mit Azure DevOps werden automatisierte Continuous Integration und Continuous Deployment (CI/CD) umgesetzt, um Entwicklungs- und Bereitstellungsprozesse zu optimieren. Aktuelle Veröffentlichungen zeigen, dass CI/CD-Pipelines die Bereitstellungszeit um bis zu 50 \% reduzieren können \cite{azureDevOps, fowler2020continuous}.
     \item \textbf{HBS-Masstransit Artefakt:} Mit Azure DevOps werden automatisierte Continuous Integration und Continuous Deployment (CI/CD) umgesetzt, um Entwicklungs- und Bereitstellungsprozesse zu optimieren. Aktuelle Veröffentlichungen zeigen, dass CI/CD-Pipelines die Bereitstellungszeit um bis zu 50 \% reduzieren können \cite{azureDevOps, fowler2020continuous}.

      \item \textbf{Repository:} Mit Azure DevOps werden automatisierte Continuous Integration und Continuous Deployment (CI/CD) umgesetzt, um Entwicklungs- und Bereitstellungsprozesse zu optimieren. Aktuelle Veröffentlichungen zeigen, dass CI/CD-Pipelines die Bereitstellungszeit um bis zu 50 \% reduzieren können \cite{azureDevOps, fowler2020continuous}.
\end{itemize}


\subsection*{Integration der Technologien}
Die Kombination aus React und TypeScript ermöglicht die Entwicklung einer flexiblen und responsiven Benutzeroberfläche. Backend-seitig sorgt .NET Core für eine skalierbare und performante Architektur, während MS SQL als sichere Speicherlösung dient. Die enge Verzahnung dieser Technologien gewährleistet eine effiziente Datenverarbeitung und nahtlose Integration in bestehende Infrastrukturen \cite{microsoftAzure}.

\section{Zusammenfassung}
Mitarbeitendengespräche, unterstützt durch Datenvisualisierung und moderne Technologien, bieten erhebliches Potenzial zur Förderung der Teamdynamik und zur Erreichung von Unternehmenszielen. Die theoretischen Grundlagen zeigen, dass die Kombination von strukturierten Gesprächen, datenbasierter Entscheidungsfindung und technologischen Innovationen die Basis für ein effizientes und skalierbares System bildet.

