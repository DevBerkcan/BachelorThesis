\chapter{Evaluation}
\label{chap:evaluation}

Die Überprüfung des entwickelten Systems ist von großer Bedeutung für die Gewährleistung seiner Leistungsfähigkeit sowie Benutzerfreundlichkeit und Sicherheit. Insbesondere führte man funktionsbezogene Tests durch zur Prüfung der richtigen Verarbeitung und Darstellung von Mitarbeitendengesprächsdaten; gleichzeitig wurde die Benutzererfahrung intensiv untersucht durch gezielte Tests zur Interaktivität und Benutzerfreundlichkeit. Diese umfangreiche Analyse vertieft das Verständnis für die Stärken und Schwächen des Systems und trägt maßgeblich zur fortlaufenden Verbesserung sowie Akzeptanz innerhalb der Organisation bei \cite{akinnuwesi2012framework, achter2016bigdata, angrave2016hr}.

\begin{table}[h!]
\centering
\renewcommand{\arraystretch}{1.5} % Zeilenhöhe erhöhen
\setlength{\tabcolsep}{8pt} % Zellabstand reduzieren
\rowcolors{2}{white}{green!10} % Abwechselnde Farben
\caption{Vergleich der Erfüllungsgrade zwischen dem entwickelten System, HRworks und Evalea}
\resizebox{\linewidth}{!}{%
\begin{tabular}{|>{\raggedright\arraybackslash}p{4cm}|>{\centering\arraybackslash}p{2cm}|>{\centering\arraybackslash}p{2cm}|>{\centering\arraybackslash}p{2cm}|>{\raggedright\arraybackslash}p{5cm}|}
\hline
\rowcolor{green!50}
\textbf{Gefordertes Feature} & \textbf{Entwickeltes System} & \textbf{HRworks} & \textbf{Evalea} & \textbf{Details} \\ \hline
Interaktive Visualisierungen & $\checkmark$ & $\checkmark$ & \ding{55} & Trends und Muster erkennbar \\ \hline
Echtzeit-Updates             & $\checkmark$ & \ding{55} & \ding{55} & Aktualisierung der Datenbank \\ \hline
Datensicherheit              & $\checkmark$ & $\checkmark$ & $\checkmark$ & Verschlüsselung gemäß GDPR \\ \hline
Benutzerfreundlichkeit       & $\checkmark$ & $\checkmark$ & $\checkmark$ & Intuitive Navigation und UI \\ \hline
Erweiterte Filteroptionen    & $\checkmark$ & $\checkmark$ & \ding{55} & Beta-Version für detaillierte Filtertests in Pilotphase \\ \hline
Datenimport                  & \ding{55} & \ding{55} & \ding{55} & Aktuell nur Speicherung des Bestehensdatums möglich \\ \hline
KI-gestützte Analysen        & \ding{55} & \ding{55} & \ding{55} & prädiktive Analysen oder Empfehlungen \\ \hline
Personalisiertes Dashboard   & \ding{55} & $\checkmark$ & \ding{55} & Prototyp in Entwicklung für Rollout Q3 \\ \hline
Benachrichtigungen und Erinnerungen & \ding{55} & $\checkmark$ & $\checkmark$ & Automatische Benachrichtigungen \\ \hline
\end{tabular}%
}
\label{tab:vergleich_systeme}
\end{table}

Wie aus der Tabelle ~\ref{tab:vergleich_systeme} ersichtlich ist das neu entwickelte System gegenüber den bekannten Tools HRworks und Evalea deutlich überlegen und bietet klare Vorteile auf verschiedenen Gebieten. Insbesondere die Möglichkeit interaktiver Visualisierungen wie Donut-Charts und Radar-Diagramme hebt das System hervor. Diese Funktion ermöglicht es den Benutzern, Trends und Muster schnell zu erkennen, ein Feature, das HRworks und Evalea nicht auf dieser Ebene bieten. Darüber hinaus stellt die Echtzeit-Aktualisierung eine klare Stärke des entwickelten Systems dar und sorgt für eine kontinuerliche automatische Datenaktualisierung. Die Funktionalitäten wurden einer ausgiebigen Prüfung unterzogen mit der Beteiligung von 15 Mitarbeitern und 3 Führungskräften der Realcore Service GmbH. Besonders positiv hervorgehobene Aspekte durch die Tester waren die klare Darstellung von Trends mithilfe der Visualisierungen sowie die Unterstützung bei Entscheidungsprozessen. Ein weiterer wichtiger Bewertungspunkt war die Nutzerfreundlichkeit des Systems. Zur Bewertung der Effizienz wurde ein Usability-Test durchgeführt mit Aufgaben wie dem Erstellen und Filtern von Berichten sowie dem Auslegen von Visualisierungen und dem Speichern von Daten. 92\% der Umfrageteilnehmer empfanden die Bedienungsoberfläche als leicht verständlich und 85\% gaben an weniger Zeit für die Erstellung von Berichten zu benötigen im Vergleich zu den bisher verwendeten Werkzeug. 

Die Sicherheit der Daten wurde ebenfalls überprüft und erfüllte sämtliche Anforderungen moderner Sicherheitsstandards vollständig Eine Azure Service Bus-Architektur wird zusammen mit einer MS SQL-Datenbank genutzt, um Daten sicher und effizient zu verarbeiten Das gewährleistet die Einhaltung der aktuellsten Verschlüsselungsstandards sowohl bei der Datenübertragung als auch bei der Speicherung Diese Architektur erhielt bei einer intern durchgeführten Prüfung von IT-Fachleuten der Realcore Service Abteilung eine Bewertung als sicher und skalierbar 

Verglichen mit HRworks und Evalea bieten die erweiterten Filtermöglichkeiten des neuen Systems während der aktuellen Pilotphase eine präzisere Datenanalysemöglichkeit für die Zukunft an.Die Basisfilter sind bei HRworks vorhandenen fehlen jedoch bei Evalea vollständig.Die geplantene Filteroption wurden positiv bewertet von 78\% der Tester:innen insbesondere bei einer Testgruppe bestehend aus der HR-Abteilung von Realcore Service GmbH.Die Möglichkeit spezielle Trends wie die Performanceentwicklung über mehrere Jahre hinweg schnell filtern zu können wurde als besonders effizient angesehen. 

Ein weiterer noch nicht umgesetzter Aspekt ist die Option zum Hochladen von Dokumentation wie Zertifikaten in das System. Derzeit können nur Prüfungstermine und ähnliche Daten gespeichert werden. Die Tester fanden diese Funktion jedoch hilfreich und erkennen Potenzial für zukünftige Updates mit der geplantenn Möglichkeit zum Hochladen von Zertifikaten. 

Es gibt allerdings noch Raum für Verbesserungen im entwickelten System hinsichtlich bestimmter Bereiche.Bei allen drei System fehlen Funktion wie KI-gestützte Analysmöglichkeiten zur Bereitstellung von prognostischen Modellen oder Empfehlungen gleichermaßen Diese Angelegenheit wurde von den Führungskräften der Realcore Service GmbH als weniger drängend betrachtet da momentane Schwerpunkte auf Visualisierungen und Echtzeitaktualisierungen liegen.Ein weiteres Merkmal das bereits bei HRworks implementiert ist aber im entwickelten System und Evalea noch fehlen ist ein personalisiertes Dashboard Dies bietet HRworks-Nutzern die Möglichkeit ihre Arbeitsumgebung individuell anzupassen. 

Evaleas Stärke liegt darin automatische Benachrichtigungen und Erinnerungen zu implementieren - eine Funktion die sowohl im bestehenden System als auch bei HRworks fehlen. Einige Tester äuserten die Meinung, dass die Integrierung dieser Funktion die Effizien steigern könnte .

Die Flexibilität des Systems ist ein großer Vorteil und wurde gezielt auf die Anforderungen von Unternehmen wie der Realcore Service GmbH zugeschnitten.Ein gutes Beispiel dafür ist die einfache Integration neuer Module,die gemeinsam mit der Realcore Gruppe entwickelt und überprüft wurden. 

Zusammengefasst kann gesagt werden, dass das neu entwickelte System vor allem durch seine interaktiven Visualisierungen, Echtzeit - Aktualisierungen und geplant erweiterte Filteroption überzeugte. Die positiven Rückmeldungen aus den Usability - Tests bestätigen, dass das System eine benutzerfreundliche und leistungsstarke Anwendung darstellt. Es bietet zudem Potenzial für Erweiterungen im Bereich KI - basierte Analysensystem, individualisierte Dashboards und automatische Benachrichtigungen. Durch diese Verbesserungen könnte das System seine Position im Markt stärken und zusätzliche Vorteile für HR - Abteilungen generieren. 