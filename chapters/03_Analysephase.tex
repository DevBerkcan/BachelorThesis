\chapter{Analysephase}
\label{chap:analysephase}

\section{Anforderungen}
\subsection{Funktionale Anforderungen}
Das Tool soll folgende Funktionen bieten:
\begin{itemize}
    \item \textbf{Visualisierung:} Darstellung der Gesprächsdaten als Donut- und Radarcharts.
    \item \textbf{Datenmanagement:} Import, Speicherung und Bearbeitung von Gesprächsdaten.
    \item \textbf{Export:} Generierung von Berichten in PDF-Format.
    \item \textbf{Benutzerverwaltung:} Rollenspezifischer Zugriff (z. B. Admins, HR, Führungskräfte).
\end{itemize}

\subsection{Nicht-funktionale Anforderungen}
\begin{itemize}
    \item \textbf{Performance:} Schnelle Verarbeitung großer Datenmengen.
    \item \textbf{Skalierbarkeit:} Anpassung an wachsende Nutzer- und Datenmengen.
    \item \textbf{Sicherheit:} Schutz sensibler Daten durch Authentifizierung und Verschlüsselung.
\end{itemize}

\section{Stakeholder-Interviews und Umfragen}
Durchführung von Interviews mit Führungskräften und HR-Mitarbeitenden zur Bedarfsanalyse.  
Ergebnisse:
\begin{itemize}
    \item Bedarf an intuitiven Diagrammen, die Trends und Muster visualisieren.
    \item Wunsch nach Echtzeit-Updates und übersichtlicher Benutzeroberfläche.
    \item Anforderung an sichere Speicherung und Vertraulichkeit der Daten.
\end{itemize}

\section{Vergleich bestehender Lösungen}
Tools wie Tableau und Power BI:
\begin{itemize}
    \item \textbf{Stärken:} Leistungsstarke Visualisierungen, breite Funktionalität.
    \item \textbf{Schwächen:} Hoher Preis, oft komplizierte Bedienung.
\end{itemize}

Identifizierte Lücken:
\begin{itemize}
    \item Mangelnde Anpassung an spezifische Anforderungen von Mitarbeitendengesprächen.
    \item Keine Integration in bestehende HR-Prozesse.
\end{itemize}