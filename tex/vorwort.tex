\chapter*{Vorwort}
\addcontentsline{toc}{chapter}{Vorwort} % Fügt das Vorwort ins Inhaltsverzeichnis ein

Die vorliegende Bachelorarbeit entstand im Rahmen meines Studiums der Informatik (B.Sc.) an der Hochschule Bochum. Das Thema „Analyse und Entwicklung eines Systems zur Visualisierung von Mitarbeitendengesprächsdaten“ wurde gewählt, da es sowohl ein spannendes und praxisnahes Forschungsfeld darstellt als auch direkt mit einer von mir entwickelten Anwendung verbunden ist. Diese Anwendung entstand ursprünglich aus einem konkreten Bedarf in der Praxis und wurde im Laufe der Arbeit kontinuierlich weiterentwickelt und verfeinert, um den wissenschaftlichen und praktischen Anforderungen gerecht zu werden.

Die Entwicklung dieses Systems bot nicht nur einen tiefen Einblick in moderne Technologien und deren Anwendung im HR-Bereich, sondern auch die Möglichkeit, praxisnahe Lösungen für komplexe Probleme zu erarbeiten. Besonders die Verbindung von theoretischem Wissen mit praktischen Umsetzungen stellte einen zentralen Lerngewinn dar.

Die Arbeit unterstreicht die Bedeutung einer interdisziplinären Herangehensweise, bei der technologische Innovationen auf die Bedürfnisse der Arbeitswelt abgestimmt werden. Moderne HR-Management-Systeme können nicht nur die Effizienz von Prozessen steigern, sondern auch die Qualität von Entscheidungen verbessern und eine transparente Kommunikation zwischen Führungskräften und Mitarbeitenden fördern. 

Abschließend verdeutlicht diese Arbeit, wie moderne Technologien und datenbasierte Visualisierungen die Praxis von Mitarbeitendengesprächen nachhaltig transformieren können. Die gewonnenen Einsichten und praktischen Erfahrungen bieten eine solide Grundlage, um diese Konzepte in zukünftigen Projekten und beruflichen Aufgabenfeldern weiterzuentwickeln. Damit leistet die Arbeit einen relevanten Beitrag zur Weiterentwicklung moderner HR-Strategien und bietet zugleich eine fundierte Basis für zukünftige Forschung und Praxis.

Mein besonderer Dank gilt meiner Erstprüferin, Frau Anja Tenberge, für ihre konstruktive Unterstützung und hilfreichen Anregungen. Ebenso danke ich Herrn Sascha Bajonczak als Zweitprüfer sowie allen Kolleginnen und Kollegen, die mich während der Erstellung dieser Arbeit unterstützt haben. Nicht zuletzt möchte ich meiner Familie und meinen Freunden danken, die mir durch ihre Unterstützung in dieser intensiven Zeit stets den Rücken gestärkt haben.

\vspace{1cm}

\begin{flushright}
Düsseldorf, Januar 2025 \\
Berk-Can Atesoglu
\end{flushright}
