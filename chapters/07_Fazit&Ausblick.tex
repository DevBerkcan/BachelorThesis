\chapter{Fazit und Ausblick} 
\label{chap:fazit}

\section{Zusammenfassung der Ergebnisse}
Die vorliegende wissenschaftliche Arbeit widmete sich der Entwicklung eines Systems zur Visualisierung von Mitarbeitendengespr"achdaten mit dem Ziel, datengetriebene Entscheidungsprozesse im HR-Kontext zu unterst"utzen und zu optimieren. Dieses Ziel wurde durch die eingehende Untersuchung moderner Technologien, datenbasierter Visualisierungen und einer benutzerzentrierten Systemarchitektur erreicht. Besonderer Fokus lag auf der benutzerfreundlichen, effizienten und sicheren Gestaltung des Systems, um sowohl F"uhrungskr"afte als auch Mitarbeitende bei der Analyse und Auswertung zu unterst"utzen.

Die Arbeit zeigte, dass durch den Einsatz moderner Technologien wie React, .NET Core und Microsoft Azure eine robuste Basis f"ur die Entwicklung eines flexiblen und skalierbaren Systems geschaffen werden konnte. Die Integration interaktiver Visualisierungstools wie Donutcharts und Radarcharts erwies sich als besonders effektiv, um komplexe Daten anschaulich und verst"andlich darzustellen. Diese Visualisierungen f"orderten nicht nur eine vereinfachte Dateninterpretation, sondern erm"oglichten auch die gezielte Ableitung strategischer Ma\ss nahmen auf Basis fundierter Entscheidungsgrundlagen. 

Das System erm"oglicht es zudem den Mitarbeitenden, ihre eigene Leistung in einem dedizierten Bereich zu "uberwachen. Dort k"onnen sie zwei Gespr"achsjahre miteinander vergleichen, um ihre Fortschritte und Entwicklungsbereiche im "Uberblick zu behalten. Dar"uber hinaus bietet das System die Funktion, Zertifikate der Mitarbeitenden im Azure Blob Storage hochzuladen und sicher zu speichern. Dies gew"ahrleistet eine zentrale und zugriffssichere Ablage wichtiger Dokumente.

Dar"uber hinaus belegten die theoretischen Grundlagen und technologischen Analysen, dass datenbasierte Ans"atze und Visualisierungen weit "uber die reine Darstellung von Informationen hinausgehen. Sie bieten einen klaren Mehrwert, insbesondere bei der Identifikation von Kompetenzprofilen, der Verfolgung individueller Entwicklungsziele und der Abstimmung dieser Ziele mit den strategischen Anforderungen eines Unternehmens. Das entwickelte System stellt somit einen relevanten Beitrag zur Digitalisierung von HR-Prozessen dar.

Im Kontext der Forschungsfrage konnte nachgewiesen werden, dass datenbasierte Visualisierungssysteme zur Optimierung von Mitarbeitendengespr"achen beitragen k"onnen. Die Ergebnisse der Arbeit verdeutlichen, dass ein solches System nicht nur eine objektive und transparente Entscheidungsgrundlage schafft, sondern auch die Kommunikation zwischen F"uhrungskr"aften und Mitarbeitenden verbessert. Dies f"uhrt sowohl zu einer h"oheren Zufriedenheit bei Mitarbeitenden als auch zu einer optimierten Ausrichtung von individuellen und organisatorischen Zielen. 

Gleichzeitig wurden innovative Technologien und Visualisierungskonzepte mit interaktiven Oberfl"achenelementen kombiniert, um die Akzeptanz solcher Systeme im Arbeitsalltag zu erh"ohen. Im Vergleich zu bisherigen Studien, die sich auf theoretische HR-Analytics-Ans"atze konzentrieren, bietet diese Arbeit eine praktische Perspektive und hebt die Bedeutung benutzerzentrierter Designprinzipien hervor.

\section{Potenzielle Weiterentwicklungen}
Die Arbeit hat gezeigt, dass datenbasierte Visualisierungssysteme im HR-Bereich ein gro\ss es Potenzial bieten. Dennoch gibt es zahlreiche M"oglichkeiten, das entwickelte System weiterzuentwickeln:
\begin{itemize}
    \item \textbf{Integration von Machine-Learning-Modellen:} Der Einsatz pr"adiktiver Algorithmen k"onnte dabei helfen, Muster in den Daten zu erkennen und zuk"unftige Entwicklungen, wie die Wirkung bestimmter Ma\ss nahmen, vorherzusagen.
    \item \textbf{Erweiterung auf andere HR-Prozesse:} Neben Mitarbeitendengespr"achen k"onnten auch Prozesse wie Onboarding, Schulungsplanung oder Teamanalysen von einer datenbasierten L"osung profitieren.
    \item \textbf{Internationalisierung:} Die Mehrsprachigkeit des Systems w"are eine wichtige Erweiterung, um das Tool in global t"atigen Unternehmen einzusetzen.
    \item \textbf{Integration in bestehende HR-Software:} Eine tiefere Verkn"upfung mit Tools wie HRworks oder Evalea w"urde die Akzeptanz und Effizienz des Systems erh"ohen.
    \item \textbf{Erweiterte Visualisierungsoptionen:} Die Implementierung zus"atzlicher Diagrammtypen, wie beispielsweise Bubble Charts oder Heatmaps, k"onnte die Analyseflexibilit"at weiter steigern.
    \item \textbf{Feldtests in Unternehmen:} Umfangreiche Tests in realen Anwendungsszenarien k"onnten den praktischen Nutzen validieren und m"ogliche Schw"achen aufdecken.
\end{itemize}

Diese Weiterentwicklungen unterstreichen die Relevanz und das Potenzial, das in datenbasierten HR-Systemen steckt, und bieten eine klare Perspektive f"ur zuk"unftige Forschungs- und Entwicklungsarbeiten.

\section{Pers"onlicher Ausblick}
Die Entwicklung dieses Systems bot nicht nur einen tiefen Einblick in moderne Technologien und deren Anwendung im HR-Bereich, sondern auch die M"oglichkeit, praxisnahe L"osungen f"ur komplexe Probleme zu erarbeiten. Besonders die Verbindung von theoretischem Wissen mit praktischen Umsetzungen stellte einen zentralen Lerngewinn dar.

Die Arbeit unterstreicht die Bedeutung einer interdisziplin"aren Herangehensweise, bei der technologische Innovationen auf die Bed"urfnisse der Arbeitswelt abgestimmt werden. Moderne HR-Management-Systeme k"onnen nicht nur die Effizienz von Prozessen steigern, sondern auch die Qualit"at von Entscheidungen verbessern und eine transparente Kommunikation zwischen F"uhrungskr"aften und Mitarbeitenden f"ordern. 

Abschlie\ss end verdeutlicht diese Arbeit, wie moderne Technologien und datenbasierte Visualisierungen die Praxis von Mitarbeitendengespr"achen nachhaltig transformieren k"onnen. Die gewonnenen Einsichten und praktischen Erfahrungen bieten eine solide Grundlage, um diese Konzepte in zuk"unftigen Projekten und beruflichen Aufgabenfeldern weiterzuentwickeln. Damit leistet die Arbeit einen relevanten Beitrag zur Weiterentwicklung moderner HR-Strategien und bietet zugleich eine fundierte Basis f"ur zuk"unftige Forschung und Praxis.
