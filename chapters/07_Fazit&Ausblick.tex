\chapter{Fazit und Ausblick} 
\label{chap:fazit}

Die vorliegende wissenschaftliche Arbeit widmete sich der Entwicklung eines Systems zur Visualisierung von Mitarbeitendengesprächsdaten mit dem Ziel, datengetriebene Entscheidungsprozesse im HR-Kontext zu unterstützen und zu optimieren. Dieses Ziel wurde durch die eingehende Untersuchung moderner Technologien, datenbasierter Visualisierungen und einer benutzerzentrierten Systemarchitektur erreicht. Besonderer Fokus lag auf der benutzerfreundlichen, effizienten und sicheren Gestaltung des Systems, um sowohl Führungskräfte als auch Mitarbeitende bei der Analyse und Auswertung zu unterstützen.

Die Arbeit zeigte, dass durch den Einsatz moderner Technologien wie React, .NET Core und Microsoft Azure eine robuste Basis für die Entwicklung eines flexiblen und skalierbaren Systems geschaffen werden konnte. Die Integration interaktiver Visualisierungstools wie Donutcharts und Radarcharts erwies sich als besonders effektiv, um komplexe Daten anschaulich und verständlich darzustellen. Diese Visualisierungen förderten nicht nur eine vereinfachte Dateninterpretation, sondern ermöglichten auch die gezielte Ableitung strategischer Maßnahmen auf Basis fundierter Entscheidungsgrundlagen. 

Das System ermöglicht es zudem den Mitarbeitenden, ihre eigene Leistung in einem dedizierten Bereich zu überwachen. Dort können sie zwei Gesprächsjahre miteinander vergleichen, um ihre Fortschritte und Entwicklungsbereiche im Überblick zu behalten. Darüber hinaus bietet das System die Funktion, Zertifikate der Mitarbeitenden im Azure Blob Storage hochzuladen und sicher zu speichern. Dies gewährleistet eine zentrale und zugriffssichere Ablage wichtiger Dokumente.

Darüber hinaus belegten die theoretischen Grundlagen und technologischen Analysen, dass datenbasierte Ansätze und Visualisierungen weit über die reine Darstellung von Informationen hinausgehen. Sie bieten einen klaren Mehrwert, insbesondere bei der Identifikation von Kompetenzprofilen, der Verfolgung individueller Entwicklungsziele und der Abstimmung dieser Ziele mit den strategischen Anforderungen eines Unternehmens. Das entwickelte System stellt somit einen relevanten Beitrag zur Digitalisierung von HR-Prozessen dar.

Die Stärken des entwickelten Systems liegen insbesondere in seiner Benutzerfreundlichkeit, Leistungsfähigkeit und Sicherheit. Die klare Benutzeroberfläche und intuitive Navigation ermöglichen es Nutzenden, das System effizient zu bedienen, während die schnelle Verarbeitung und Visualisierung großer Datenmengen selbst bei umfangreichen Datensätzen eine reibungslose Nutzung sicherstellt \cite{heer2012interactive, chen2012interactive}. Zudem gewährleisten zuverlässige Authentifizierungs- und Verschlüsselungsmechanismen einen hohen Sicherheitsstandard \cite{boneder2023evaluation}. Diese Eigenschaften tragen wesentlich zur Akzeptanz und Effizienz des Systems bei.

Gleichzeitig müssen jedoch auch die Schwächen kritisch reflektiert werden. Eine der zentralen Einschränkungen ist die begrenzte Erweiterbarkeit des Systems, da bislang keine Integration von KI-gestützten Analysetools erfolgt ist \cite{tambe2019artificial}. Solche Tools könnten dazu beitragen, verborgene Muster in den Daten aufzudecken und prädiktive Analysen zu ermöglichen. Zudem war die Reichweite der durchgeführten Usability-Tests limitiert, insbesondere im Hinblick auf reale Anwendungsszenarien. Diese Einschränkung erschwert eine umfassende Beurteilung der Praxistauglichkeit. Herausforderungen ergaben sich auch bei der Integration in bestehende HR-Software, insbesondere aufgrund unterschiedlicher Datenformate und proprietärer Schnittstellen. 

Im Kontext der Forschungsfrage konnte nachgewiesen werden, dass datenbasierte Visualisierungssysteme zur Optimierung von Mitarbeitendengesprächen beitragen können. Die Ergebnisse der Arbeit verdeutlichen, dass ein solches System nicht nur eine objektive und transparente Entscheidungsgrundlage schafft, sondern auch die Kommunikation zwischen Führungskräften und Mitarbeitenden verbessert. Dies führt sowohl zu einer höheren Zufriedenheit bei Mitarbeitenden als auch zu einer optimierten Ausrichtung von individuellen und organisatorischen Zielen. 

Gleichzeitig wurden innovative Technologien und Visualisierungskonzepte mit interaktiven Oberflächenelementen kombiniert, um die Akzeptanz solcher Systeme im Arbeitsalltag zu erhöhen. Im Vergleich zu bisherigen Studien, die sich auf theoretische HR-Analytics-Ansätze konzentrieren, bietet diese Arbeit eine praktische Perspektive und hebt die Bedeutung benutzerzentrierter Designprinzipien hervor.

Im Ausblick bieten sich mehrere Verbesserungspotenziale, die zukünftige Entwicklungen des Systems unterstützen könnten. Eine zentrale Erweiterung wäre die Integration von Machine-Learning-Modellen, um die Analysemöglichkeiten zu erweitern und beispielsweise verborgene Muster in den Daten zu erkennen oder prädiktive Analysen durchzuführen \cite{aral2012threeway}. Zudem sollte eine breitere Testabdeckung mit realen Unternehmensdaten angestrebt werden, um die Praxistauglichkeit des Systems weiter zu validieren und Schwachstellen frühzeitig zu identifizieren. Ein weiteres Potenzial liegt in der Optimierung der Benutzeroberfläche basierend auf weiterem Nutzerfeedback, um die Bedienbarkeit und Nutzerfreundlichkeit weiter zu verbessern \cite{sedlmair2011information}. 

Diese Weiterentwicklungen unterstreichen die Relevanz und das Potenzial, das in datenbasierten HR-Systemen steckt, und bieten eine klare Perspektive für zukünftige Forschungs- und Entwicklungsarbeiten.
