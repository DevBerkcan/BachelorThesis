\chapter{Einleitung}
\label{chap:einleitung}

\section{Motivation}
Die Qualität und Effizienz von Mitarbeitendengesprächen spielen eine entscheidende Rolle für den Erfolg eines Unternehmens. Sie bieten eine Plattform für Feedback, Zielvereinbarungen und die Identifikation von Entwicklungspotenzialen. Trotz dieser Bedeutung bleibt die Analyse und Nachverfolgung der Ergebnisse häufig unstrukturiert. Häufig werden Gesprächsdaten manuell dokumentiert, was nicht nur zeitaufwändig, sondern auch fehleranfällig ist.

Durch den Einsatz moderner Technologien zur Datenvisualisierung können jedoch Muster und Trends aus den Gesprächsdaten abgeleitet werden, die neue Einblicke und eine datenbasierte Entscheidungsfindung ermöglichen. Diese Arbeit widmet sich der Entwicklung eines Tools, das diesen Anforderungen gerecht wird.

\section{Problemstellung}
In vielen Unternehmen werden Mitarbeitendengesprächsdaten zwar erfasst, aber nicht effektiv analysiert oder visualisiert. Ein Großteil der Informationen verbleibt in statischen Berichten oder Tabellen, was eine datengetriebene Nutzung der Ergebnisse erschwert. Besonders Führungskräfte benötigen jedoch intuitive Visualisierungen, um schnell Einblicke in die Leistung, Entwicklung und Zielerreichung ihrer Teams zu erhalten.

Es fehlt an einer Lösung, die Gesprächsdaten nicht nur speichert, sondern auch interaktiv und visuell ansprechend aufbereitet, um die Analyse zu vereinfachen und fundierte Entscheidungen zu fördern.

\section{Zielsetzung}
Das Ziel dieser Arbeit ist die Entwicklung eines Reporting-Tools, das Mitarbeitendengesprächsdaten strukturiert erfasst, speichert und visuell darstellt. Das Tool soll Führungskräften die Möglichkeit bieten, Gespräche über mehrere Zeiträume hinweg zu vergleichen, Trends zu erkennen und die Leistung ihrer Teams besser zu bewerten.

Im Fokus stehen die benutzerfreundliche Visualisierung der Daten in Form von Donut-Charts und Radarcharts sowie die Sicherstellung einer hohen Skalierbarkeit und Performance.

\section{Aufbau der Arbeit}
Die Arbeit ist wie folgt aufgebaut:
\begin{itemize}
    \item Kapitel 2 behandelt die theoretischen Grundlagen und erläutert die Bedeutung von Mitarbeitendengesprächen sowie die Relevanz moderner Datenvisualisierung.
    \item Kapitel 3 analysiert die funktionalen und nicht-funktionalen Anforderungen an das System und beschreibt die Bedürfnisse der Stakeholder.
    \item Kapitel 4 beschreibt die Konzeption des Systems, einschließlich der Systemarchitektur, des Datenmodells und des Visualisierungsdesigns.
    \item Kapitel 5 widmet sich der technischen Implementierung des Tools und stellt die verwendeten Technologien, Herausforderungen und Lösungen vor.
    \item Kapitel 6 evaluiert das entwickelte System anhand von Funktionalitäts- und Usability-Tests.
    \item Kapitel 7 fasst die Ergebnisse zusammen, diskutiert potenzielle Weiterentwicklungen und gibt einen Ausblick auf die zukünftige Anwendung des Tools.
\end{itemize}
