\chapter{Evaluation}
\label{chap:evaluation}

\section{Testmethoden}
\begin{itemize}
    \item \textbf{Funktionstests:} Überprüfung, ob alle funktionalen Anforderungen erfüllt werden.
    \item \textbf{Usability-Tests:} Bewertung der Benutzerfreundlichkeit durch Testpersonen.
\end{itemize}

\section{Ergebnisse der Evaluation}
\subsection{Funktionstests}
Das System erfüllt alle Kernanforderungen (z. B. schnelle Ladezeiten, korrekte Visualisierung).

\subsection{Usability-Tests}
Hohe Zufriedenheit der Testpersonen (z. B. intuitive Bedienung, klare Visualisierungen).

\section{Vergleich mit bestehenden Tools}
Das entwickelte Tool ist spezifischer auf die Anforderungen von Mitarbeitendengesprächen zugeschnitten.  
\textbf{Vorteile:} Geringere Kosten, bessere Anpassungsmöglichkeiten.

\section{Kritische Reflexion}
\begin{itemize}
    \item \textbf{Stärken:} Benutzerfreundlichkeit, hohe Performance, intuitive Visualisierungen.
    \item \textbf{Schwächen:} Fehlende erweiterte Analysefunktionen (z. B. KI-gestützte Auswertungen).
\end{itemize}