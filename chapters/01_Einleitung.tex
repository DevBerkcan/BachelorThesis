\chapter{Einleitung}
\label{chap:einleitung}

\section{Motivation}
Die Qualität und Effizienz von Mitarbeitendengesprächen spielen eine entscheidende Rolle für den Erfolg eines Unternehmens. Diese Gespräche bieten nicht nur eine Plattform, um Rückmeldungen zur individuellen Leistung zu geben, sondern ermöglichen auch die Festlegung zukünftiger Ziele sowie die Planung der persönlichen und beruflichen Weiterentwicklung der Mitarbeitenden. Der strategische Einsatz von Mitarbeitendengesprächen unterstützt Unternehmen dabei, ihre langfristigen Ziele zu erreichen, die Motivation der Mitarbeitenden zu fördern und ihre Produktivität zu steigern \cite{schober2008}.

In einer zunehmend datengetriebenen Welt haben Unternehmen erkannt, dass Daten ein wertvolles Gut sind. Die Fähigkeit, große Mengen an Daten zu sammeln, zu analysieren und zu nutzen, hat sich zu einem entscheidenden Wettbewerbsvorteil entwickelt. Dies gilt auch für Mitarbeitendengespräche, deren Ergebnisse oft in unstrukturierter Form vorliegen. Ohne geeignete Tools zur Analyse und Visualisierung bleiben die darin enthaltenen Erkenntnisse ungenutzt. Dies stellt insbesondere in einer Zeit, in der datengetriebene Entscheidungsprozesse die Grundlage für viele geschäftliche Strategien bilden, eine verpasste Chance dar.

Moderne Technologien zur Datenvisualisierung ermöglichen es, komplexe Daten auf einfache und intuitive Weise darzustellen. Wie Kirk \cite{kirk2016data} hervorhebt, können visuelle Darstellungen dazu beitragen, Muster und Trends zu erkennen, die in rein tabellarischen Daten verborgen bleiben. Interaktive Dashboards, die benutzerfreundlich gestaltet sind, bieten Führungskräften und Personalabteilungen wertvolle Einblicke in die Entwicklung ihrer Teams und helfen dabei, datenbasierte Entscheidungen zu treffen.

\section{Problemstellung}
Trotz der wachsenden Bedeutung von Daten in der Geschäftswelt stehen viele Unternehmen vor der Herausforderung, Mitarbeitendengesprächsdaten effektiv zu nutzen. Diese Daten werden häufig in Tabellen, Textdokumenten oder anderen unstrukturierten Formaten gespeichert, was ihre Analyse erschwert. Führungskräfte und Personalabteilungen sind auf aussagekräftige Berichte angewiesen, um fundierte Entscheidungen treffen zu können. Der Mangel an geeigneten Tools führt jedoch dazu, dass wichtige Informationen ungenutzt bleiben \cite{duarte2012performance}.

Ein weiteres Problem besteht in der Fragmentierung der Datenspeicherung. Mitarbeitendengesprächsdaten werden oft dezentral und ohne klare Struktur erfasst, was ihre Integration in bestehende HR-Prozesse erschwert. Die manuelle Aufbereitung solcher Daten ist nicht nur zeitaufwändig, sondern auch anfällig für Fehler. Darüber hinaus fehlen in vielen Organisationen Werkzeuge, die es ermöglichen, diese Daten visuell ansprechend und interaktiv aufzubereiten. Dies beeinträchtigt die Fähigkeit, strategische Entscheidungen auf Grundlage verlässlicher Daten zu treffen.

\section{Zielsetzung}
Diese Arbeit hat das Ziel, ein modernes Reporting-Tool zu entwickeln, das Mitarbeitendengesprächsdaten strukturiert erfasst, speichert und visuell darstellt. Das Tool soll es ermöglichen, Daten interaktiv zu analysieren und Trends über verschiedene Zeiträume hinweg zu erkennen. Dadurch können Führungskräfte und Personalabteilungen fundierte Entscheidungen treffen und die Effizienz von HR-Prozessen steigern.

Ein Schwerpunkt liegt auf der Benutzerfreundlichkeit des Tools, das durch die Verwendung von Donut-Diagrammen und Radarcharts eine intuitive Visualisierung ermöglicht. Während Donut-Diagramme einen schnellen Überblick über den Status von Zielvereinbarungen bieten, erlauben Radarcharts eine detaillierte Analyse einzelner Leistungsdimensionen. Zusätzlich soll das System skalierbar sein, um mit wachsenden Datenmengen und sich ändernden Anforderungen Schritt zu halten \cite{ware2012information}.

\section{Aufbau der Arbeit}
Die Arbeit ist wie folgt strukturiert:
\begin{itemize}
    \item \textbf{Kapitel 2:} Behandelt die theoretischen Grundlagen, darunter die Bedeutung von Mitarbeitendengesprächen, die Relevanz von Datenvisualisierung und die eingesetzten Technologien.
    \item \textbf{Kapitel 3:} Analysiert die Anforderungen an das System, basierend auf Stakeholder-Interviews und der Untersuchung bestehender Lösungen.
    \item \textbf{Kapitel 4:} Beschreibt die Konzeption des Systems, einschließlich der Systemarchitektur, des Datenmodells und der Sicherheitskonzepte.
    \item \textbf{Kapitel 5:} Erläutert die technische Implementierung, die verwendeten Technologien und die Herausforderungen bei der Entwicklung.
    \item \textbf{Kapitel 6:} Evaluiert das System durch Funktionalitäts- und Usability-Tests und analysiert die Ergebnisse.
    \item \textbf{Kapitel 7:} Fasst die Ergebnisse zusammen, diskutiert mögliche Weiterentwicklungen und gibt einen Ausblick auf die zukünftige Anwendung des Tools.
\end{itemize}

\bibliographystyle{IEEEtran}
\bibliography{literatur}
