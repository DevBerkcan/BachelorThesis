%------------------------------------------------------------------------------
% ----- Pakete: Encoding und Sprachanpassung 
%------------------------------------------------------------------------------
\usepackage[utf8]{inputenc}				% Encoding der Keyboard-Eingabe
\usepackage[T1]{fontenc} 				% Schriftdarstellung
\usepackage[ngerman]{babel} 			% Silbentrennung, Datum, Bezeichnungen
\usepackage[german=quotes]{csquotes} 	% Anführungszeichen

%------------------------------------------------------------------------------
% ----- Pakete: Formelsatz und Einheiten
%------------------------------------------------------------------------------
\usepackage{amsmath}
\usepackage{amssymb}
\usepackage{nicefrac}
\usepackage{siunitx}
	\sisetup{exponent-product=\cdot,
			 output-decimal-marker={,},
			 separate-uncertainty=true,
			 tight-spacing=true,
			 per-mode=fraction,
			 range-phrase=--,
			 range-units=single,
			 load-configurations = abbreviations}
	\DeclareSIUnit[]\year{y}
	\DeclareSIUnit{\bit}{bit}

%------------------------------------------------------------------------------
% ----- Pakete: Schriftarten, Symbole
%------------------------------------------------------------------------------
\usepackage{marvosym, wasysym, pifont}
\usepackage{xspace}
\usepackage{setspace}
\usepackage{eurosym}
	\DeclareSIUnit{\EUR}{\text{\euro}}
	
%------------------------------------------------------------------------------
% ----- Pakete: Tabellen
%------------------------------------------------------------------------------
\usepackage{tabularx}
\usepackage{multirow}
\usepackage{booktabs}

%------------------------------------------------------------------------------
% ------ Pakete: Abbildungen, Bilder, Zeichnungen
%------------------------------------------------------------------------------
\usepackage{xcolor}
	\definecolor{BOred}{HTML}{E2001A}
	\definecolor{BOgreen}{HTML}{ADD009}
	\definecolor{BOblue}{HTML}{005680}
	\definecolor{BOgray}{HTML}{7b7c7e}
	\definecolor{BOdarkgreen}{HTML}{0c6d31}
\usepackage[innercaption]{sidecap}
	\sidecaptionvpos{figure}{t}
\usepackage{caption}
	\captionsetup[table]{labelfont=bf, format=plain, labelsep=quad, singlelinecheck=false, textfont=up, font=small}%
	\captionsetup[figure]{labelfont=bf, format=plain, labelsep=quad, singlelinecheck=false, textfont=up, font=small}%

%------------------------------------------------------------------------------
% ------ Pakete: Links, URLs
%------------------------------------------------------------------------------
\usepackage{url}
	\addto\captionsngerman{\def\figurename{Abb.}}
	\addto\extrasngerman{\def\figureautorefname{Abb.}}
	\addto\captionsngerman{\def\tablename{Tab.}}
	\addto\extrasngerman{\def\tableautorefname{Tab.}}
\usepackage[english, breaklinks]{hyperref}
\hypersetup{
    colorlinks=true,        % Aktiviert farbige Links
    linkcolor=blue,         % Farbe für interne Links (z. B. Inhaltsverzeichnis)
    citecolor=red,          % Farbe für Zitate/Referenzen
    urlcolor=blue,          % Farbe für URLs
    filecolor=magenta,      % Farbe für Dateien
    pdfborder={0 0 0},      % Deaktiviert Rahmen um Links
    linktoc=all             % Links auch im Inhaltsverzeichnis aktivieren
}


%------------------------------------------------------------------------------
% ------ Pakete: Code
%------------------------------------------------------------------------------
%\usepackage{minted, lineno}

%------------------------------------------------------------------------------
% ------ Pakete: Dummy-Text
%------------------------------------------------------------------------------
\usepackage{lipsum}

%------------------------------------------------------------------------------
% ----- Pakete: Verzeichnisse, ToDos
%------------------------------------------------------------------------------
\usepackage{titletoc}
\usepackage[backgroundcolor=red!50, linecolor=red!50, bordercolor=white]{todonotes}

%------------------------------------------------------------------------------
% ----- Pakete: Aufzählungen 
%------------------------------------------------------------------------------
\usepackage{enumitem}
	\renewcommand{\labelitemi}{$-$}
	
%------------------------------------------------------------------------------
% ----- Pakete: Seitengestaltung, Layout
%------------------------------------------------------------------------------
\usepackage[inner=3.3cm, outer=4.6cm, bottom=5.8cm, top=3.5cm]{geometry}
\usepackage{titlesec}
	\titleformat{\chapter}[hang]{\Huge}{\thechapter\hspace{20pt}\textcolor{gray}{|}\hspace{20pt}}{0pt}{\Huge}%
	\titleformat{\section}[hang]{\LARGE}{\thesection\hspace{20pt}}{0pt}{\LARGE}
	\titleformat{\subsection}[hang]{\Large}{\thesubsection\hspace{20pt}}{0pt}{\Large}%
	\titleformat{\subsubsection}[hang]{\large}{\thesubsubsection\hspace{20pt}}{0pt}{\large}%
	\titleformat*{\subsubsection}{\normalfont}
	
	% paragraph settings
	\renewcommand{\baselinestretch}{1.2}
	\setlength{\parindent}{0cm}
	
%------------------------------------------------------------------------------
% ----- Pakete: Literatur
%------------------------------------------------------------------------------
\usepackage[backend=bibtex, style=ieee]{biblatex}
\addbibresource{bib/articles.bib}

%------------------------------------------------------------------------------
% ----- Eigene Befehle
%------------------------------------------------------------------------------
	\newcommand{\zB}{\mbox{z.\,B.}\xspace}
	\newcommand{\dash}{\mbox{d.\,h.}\xspace}
	\newenvironment{smatrix}{\ensuremath\left(\begin{smallmatrix}}{\end{smallmatrix}\right)}

%------------------------------------------------------------------------------
% ----- Diagramme
%------------------------------------------------------------------------------
\usepackage{pgfplots}
\pgfplotsset{compat=1.18} % Verwende die neueste Version für optimale Ergebnisse
