\chapter{Einleitung} \label{chap:einleitung}

Die Qualität und Effizienz von Mitarbeitendengesprächen sind essenziell für den Erfolg eines Unternehmens. Solche Gespräche schaffen eine Plattform für Feedback, Leistungsbewertungen, die gemeinsame Festlegung von Zielen und die Planung individueller Entwicklungsmöglichkeiten. Regelmäßige  Gespräche tragen nicht nur zur Stärkung der Mitarbeitendenbindung bei, sondern fördern auch die Produktivität und die strategische Ausrichtung des Unternehmens, indem sie die Ziele der Organisation mit den Bedürfnissen der Mitarbeitenden in Einklang bringen \cite{Schober2008, Bryson2011}.

In einer zunehmend datengetriebenen Welt wird es immer wichtiger, große Datenmengen nicht nur zu sammeln, sondern auch sinnvoll zu nutzen. Gerade im Bereich Human Resources (\hyperref[abkuerzungen]{HR}) bleiben die Ergebnisse von Mitarbeitendengesprächen oft ungenutzt, da sie in unstrukturierter Form vorliegen. Dadurch entgehen Unternehmen wertvolle Einblicke, die zur Förderung der Mitarbeitendenentwicklung und zur Optimierung von HR-Strategien beitragen könnten \cite{Kirk2016}. Der Einsatz moderner Visualisierungstools bietet hier eine Lösung, indem er die Darstellung und Analyse solcher Daten erleichtert und datenbasierte Entscheidungen unterstützt.

Trotz der zunehmenden Digitalisierung in Unternehmen existiert eine Forschungslücke in der Verbindung zwischen datengetriebener Analyse und der praktischen Umsetzung in Mitarbeitendengesprächen. Während Studien die Vorteile datenbasierter Ansätze im (\hyperref[abkuerzungen]{HR})-Management betonen \cite{Evergreen2016}, fehlen praxisnahe Beispiele, die moderne Technologien wie interaktive Visualisierungen und cloudbasierte Datenverarbeitung zur Optimierung solcher Gespräche nutzen. Diese Arbeit schließt diese Lücke, indem sie ein System entwickelt, das datengetriebene Erkenntnisse für Mitarbeitendengespräche praktisch anwendbar macht.

Die Inhalte der Arbeit wurden in Zusammenarbeit mit der Realcore Services GmbH entwickelt, einem führenden IT-Dienstleister, der sich auf innovative Lösungen für die Optimierung interner Prozesse spezialisiert hat. Ziel der Arbeit ist es, ein modernes und benutzerfreundliches Reporting-Tool zu entwickeln, das speziell für die Analyse und Visualisierung von Daten aus Mitarbeitendengesprächen konzipiert ist. Dabei werden technologische Ansätze wie React, .NET Core und Microsoft Azure eingesetzt, um eine flexible, skalierbare und sichere Plattform zu schaffen.

Die zentrale Forschungsfrage dieser Arbeit lautet: \begin{quote} Wie können datenbasierte Visualisierungen und moderne Technologien zur Optimierung von Mitarbeitendengesprächen beitragen und deren Effizienz sowie strategische Relevanz für Unternehmen erhöhen? \end{quote}

Wissenschaftlich leistet diese Arbeit einen Beitrag zur Integration von datengetriebenen Visualisierungen und modernen Technologien in den Kontext von Mitarbeitendengesprächen. Sie untersucht nicht nur die technische Machbarkeit solcher Systeme, sondern auch die Akzeptanz und Praxistauglichkeit in der realen Arbeitswelt. Dadurch liefert sie wertvolle Erkenntnisse für zukünftige Forschung und Entwicklung in diesem Bereich.

Die zentralen Herausforderungen liegen dabei in der effizienten Verarbeitung von Daten aus Mitarbeitendengesprächen, die oft in fragmentierter oder unstrukturierter Form vorliegen. Ziel ist es, diese Daten zu organisieren und durch interaktive Visualisierungen wie Donut-Diagramme und Radarcharts zugänglich zu machen, um Führungskräfte und (\hyperref[abkuerzungen]{HR})-Abteilungen bei der Entscheidungsfindung zu unterstützen \cite{Evergreen2016}. Neben der technischen Umsetzung wird auch untersucht, wie das entwickelte System durch benutzerzentrierte Gestaltung Akzeptanz und Praxistauglichkeit erhöhen kann.

Die vorliegende Arbeit gliedert sich wie folgt: Im Kapitel~\ref{chap:theoretische-grundlagen} werden die theoretischen Grundlagen vorgestellt, die die Bedeutung von Mitarbeitendengesprächen, die Rolle der Datenvisualisierung und die eingesetzten Technologien umfassen. Kapitel~\ref{chap:analysephase} widmet sich der Analyse der Anforderungen durch Stakeholder-Interviews und einen Vergleich bestehender Lösungen. Die Konzeption des Systems, einschließlich der Systemarchitektur, des Datenmodells und der Sicherheitsmaßnahmen, wird im Kapitel~\ref{chap:konzeption} beschrieben. Kapitel~\ref{chap:implementierung} behandelt die technische Implementierung, wobei auf Herausforderungen und deren Lösungen eingegangen wird. Im Kapitel~\ref{chap:evaluation} wird das System evaluiert, indem Funktionstests und Usability-Analysen durchgeführt werden. Abschließend fasst Kapitel~\ref{chap:fazit} die Ergebnisse zusammen und gibt einen Ausblick auf mögliche Weiterentwicklungen.

Diese Arbeit soll insgesamt nicht nur zur Effizienzsteigerung von Mitarbeitendengesprächen beitragen, sondern auch zeigen, wie datenbasierte Ansätze die strategische Ausrichtung und Entscheidungsfindung in Unternehmen nachhaltig unterstützen können. Darüber hinaus liefert sie praktische Erkenntnisse, wie innovative Technologien die Digitalisierung von (\hyperref[abkuerzungen]{HR})-Prozessen vorantreiben können.