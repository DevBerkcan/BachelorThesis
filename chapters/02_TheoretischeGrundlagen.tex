\chapter{Theoretische Grundlagen}
\label{chap:theoretische-grundlagen}

Die individuelle Entwicklung von Mitarbeitenden erfolgt maßgeblich durch zielgerichtete Gespräche und datengestützte Feedbackprozesse. Die folgenden Abschnitte beleuchten die Bedeutung von Mitarbeitendengesprächen, die Rolle der Datenvisualisierung in diesem Kontext sowie die technologischen Grundlagen für deren Implementierung. Hierbei wird auf die Verknüpfung von strategischen Unternehmenszielen und persönlichen Entwicklungsmöglichkeiten eingegangen, um eine nachhaltige Verbesserung der Teamdynamik und der Leistungseffizienz zu fördern. Diese theoretischen Grundlagen bilden das Fundament für die anschließende Analyse und Entwicklung eines optimalen Systems zur Visualisierung von Mitarbeitendengesprächsdaten.

\section{Mitarbeitendengespräche und ihre Bedeutung}
Mitarbeitendengespräche, auch „MAG“ genannt, umfassen alle Gespräche, die Vorgesetzte aus verschiedenen Anlässen mit ihren Mitarbeitenden führen. Ziel dieser Gespräche ist es, Informationen zu vermitteln, einen Meinungsaustausch zu ermöglichen sowie Wertschätzung, Lob und gegebenenfalls Kritik auszudrücken \cite{schober2008}. Darüber hinaus bieten sie die Gelegenheit, Probleme zu analysieren und bevorstehende Veränderungen zu besprechen. Regelmäßige Mitarbeitendengespräche, die in der Regel ein- bis zweimal jährlich stattfinden, tragen wesentlich zur Verbesserung der innerbetrieblichen Kommunikation, Zusammenarbeit und Partizipation bei. Dabei ist die kommunikative Kompetenz der Führungskräfte eine wesentliche Grundlage für den Erfolg solcher Gespräche.

Typischerweise umfassen Mitarbeitendengespräche mehrere zentrale Elemente:
\begin{itemize}
    \item \textbf{Bewertung der bisherigen Zielerreichung:} Eine retrospektive Betrachtung, in der der Fortschritt in Bezug auf zuvor gesetzte Ziele analysiert wird \cite{duarte2012performance}.
    \item \textbf{Setzen neuer Ziele:} Definition von SMART-Zielen (spezifisch, messbar, akzeptiert, realistisch, terminiert), die die Grundlage für die zukünftige Arbeit bilden \cite{duarte2012performance}.
    \item \textbf{Diskussion von Entwicklungsmaßnahmen:} Identifikation von Schulungsbedarf, Weiterbildungsangeboten und anderen Maßnahmen, die das persönliche und berufliche Wachstum fördern \cite{bryson2011employee}.
\end{itemize}

Eine strukturierte Gesprächsführung kann die Effektivität von Führungskräften erheblich steigern, indem sie ihre Rolle als Unterstützer*innen und Entwicklungsbegleiter*innen verstärkt. Dies schließt die Bewertung von Leistungen ebenso ein wie die Identifikation gemeinsamer Entwicklungsziele, wodurch nicht nur die Bindung der Mitarbeitenden an das Unternehmen erhöht wird, sondern auch ihre beruflichen Ambitionen im Einklang mit den Unternehmenszielen stehen \cite{barton2012}. Subjektive Beurteilungen stellen jedoch eine bekannte Herausforderung in Mitarbeitendengesprächen dar, da sie zu Verzerrungen führen können. Der Einsatz datenbasierter Methoden minimiert solche Verzerrungen und schafft eine objektive Grundlage für fundierte Entscheidungen \cite{heikkila2018}.

\section{Relevanz von Datenvisualisierung}
Datenvisualisierung ist ein wesentlicher Bestandteil moderner Analytik und spielt eine entscheidende Rolle bei der Interpretation komplexer Informationen. Sie ermöglicht es, große Datenmengen in visuelle Formate umzuwandeln, die einfacher zu verstehen und zu analysieren sind \cite{kirk2016data}. Im Kontext von Mitarbeitendengesprächen kann die Visualisierung dazu beitragen, Muster und Zusammenhänge aufzudecken, die durch reine Text- oder Tabellenform schwer erkennbar sind.

Einige der Hauptvorteile von Datenvisualisierung umfassen:
\begin{itemize}
    \item \textbf{Verbesserte Entscheidungsfindung:} Führungskräfte können fundierte Entscheidungen treffen, da sie die Daten auf einen Blick verstehen \cite{kirk2016data}.
    \item \textbf{Erkennung von Trends und Abweichungen:} Historische und aktuelle Daten können leicht verglichen werden, um positive oder negative Entwicklungen zu identifizieren \cite{ware2012information}.
    \item \textbf{Kommunikation und Präsentation:} Datenvisualisierung erleichtert die Präsentation von Ergebnissen und fördert eine klare Kommunikation zwischen Teams \cite{evergreen2016effective}.
\end{itemize}

Besonders Radarcharts und Donutcharts sind für die Visualisierung von Mitarbeitendengesprächsdaten ideal geeignet. Radarcharts ermöglichen einen direkten Vergleich mehrerer Leistungsdimensionen wie Teamarbeit, Effizienz und Pünktlichkeit \cite{heikkila2018}, während Donutcharts einen schnellen Überblick über den Status von Zielvereinbarungen bieten \cite{evergreen2016effective}.

\section{Technologische Basis}
Die technologische Basis eines modernen Systems zur Visualisierung von Mitarbeitendengesprächsdaten stellt eine wesentliche Grundlage für dessen Funktionalität und Benutzerfreundlichkeit dar. Für die Entwicklung wurden moderne Technologien und Frameworks ausgewählt:

\begin{itemize}
    \item \textbf{React:} Eine JavaScript-Bibliothek, die die komponentenbasierte Entwicklung interaktiver Benutzeroberflächen ermöglicht und dabei Flexibilität sowie Skalierbarkeit bietet \cite{stefanov2021react}.
    \item \textbf{TypeScript:} Eine Erweiterung von JavaScript, die statische Typisierung bietet und dadurch die Codequalität verbessert \cite{typeScriptDocumentation}.
    \item \textbf{.NET Core:} Ein plattformunabhängiges Framework für die Entwicklung skalierbarer Webanwendungen \cite{microsoftDotNet}.
    \item \textbf{Azure SQL:} Eine cloudbasierte Datenbanklösung, die hohe Verfügbarkeit und Sicherheit bietet \cite{azureDocumentation}.
\end{itemize}

Diese Technologien gewährleisten eine nahtlose Integration und Verarbeitung großer Datenmengen. Die Kombination aus React und TypeScript ermöglicht eine dynamische Benutzeroberfläche, während .NET Core und Azure SQL die Backend-Architektur skalierbar und sicher gestalten \cite{microsoftAzure}.

Zusammenfassend zeigt sich, dass Mitarbeitendengespräche, unterstützt durch moderne Technologien und Datenvisualisierung, ein enormes Potenzial für die Förderung der Teamdynamik und die Erreichung organisatorischer Ziele bieten. Die ausgewählten Technologien schaffen eine robuste Basis für die Entwicklung eines benutzerfreundlichen Systems zur effizienten Verwaltung und Analyse von Mitarbeitendendaten.

