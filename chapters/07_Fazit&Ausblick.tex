\chapter{Fazit und Ausblick} 
\label{chap:fazit}

Das Thema der vorliegenden wissenschaftlichen Arbeit war die Schaffung eines Systems.

Das Ziel wurde durch die gründliche Analyse zeitgemäßer Technologien und datengesteuerter Maßnahmen erreicht. Visualisierungen und eine Systemarchitektur, die auf die Bedürfnisse der Benutzer ausgerichtet ist, wurden realisiert, insbesondere: Die Priorität lag auf einem nutzerfreundlichen und sicheren Design mit hoher Effizienz. Systeme zur Unterstützung von Managern und Teammitgliedern bei der Analytik und Auswertung \cite{heer2012interactive, chen2012interactive}.

Die Studie hat gezeigt, dass mit der Verwendung neuer Technologien wie React, .NET Core und Microsoft Azure eine solide Grundlage für die Erstellung eines anpassungsfähigen und flexiblen Systems geschaffen werden konnte \cite{boneder2023evaluation}.

Visualisierungswerkzeuge wie Donut-Diagramme und Radardiagramme haben sich als äußerst wirksam erwiesen, um komplexe Daten auf eine verständliche und anschauliche Weise zu präsentieren. Visualisierungen werden eingesetzt, um nicht nur die Auslegung von Daten zu vereinfachen, sondern auch die Entwicklung klug überlegter Handlungsstrategien auf der Grundlage fundierter Entscheidungen zu ermöglichen \cite{tambe2019artificial}.

Das System erlaubt den Mitarbeitenden außerdem, ihre eigene Leistung zu messen, aufzuzeichnen, Fortschritte zu vergleichen und Entwicklungsbereiche im Auge zu behalten. Im Azure Blob Storage können Daten hochgeladen und sicher aufbewahrt werden, um die Datensicherheit zu gewährleisten. Die zentrale und sicher zugängliche Aufbewahrung wichtiger Unterlagen bietet zusätzliche Vorteile \cite{aral2012threeway}.

Darüber hinaus untermauerten die theoretischen Prinzipien und technologischen Analysen die Bedeutung datenbasierter Ansätze. Sie gehen über die Bereicherung von Wissen hinaus und liefern einen transparent erkennbaren Zusatznutzen \cite{sedlmair2011information}.

Die Automatisierung von Personalprozeduren ist relevant. Die Vorzüge des erstellten Systems zeigen sich vor allem in seiner Benutzerfreundlichkeit. Die Benutzernavigation erleichtert die effiziente Nutzung des Systems und ermöglicht eine schnelle Verarbeitung und Darstellung großer Datenmengen. Dabei wird sichergestellt, dass die Verwendung großer Datensätze reibungslos funktioniert und zuverlässig bleibt \cite{burnett2021future}.

Eine der Hauptbeschränkungen ist die begrenzte Erweiterbarkeit des Systems bis jetzt, da keine Integration von Analysetools, die auf Künstlicher Intelligenz basieren, vorgenommen wurde \cite{tambe2019artificial}. Dies erschwert eine gründliche Einschätzung der Anwendbarkeit, insbesondere aufgrund von Integrationsproblemen bei der Einbindung in bereits vorhandene HR-Lösungen.

Im Zusammenhang mit der Forschungsfrage wurde festgestellt, dass datenbasierte Systeme nicht nur eine objektive und transparente Grundlage für Entscheidungen schaffen, sondern auch die Kommunikation fördern. Dies trägt zu einer gesteigerten Mitarbeiterzufriedenheit und einer optimierten Arbeitsumgebung bei \cite{boneder2023evaluation}.

Im Hinblick auf die Zukunft gibt es verschiedene Möglichkeiten zur Verbesserung für kommende Entwicklungen. Eine mögliche Erweiterung des Systems könnte die Einbindung von Unterstützungsfunktionen sein, was eine entscheidende Verbesserung darstellen würde. 

Das System soll durch Anwendungstests weiter auf seine Alltagstauglichkeit geprüft werden. Frühzeitiges Erkennen von Problemen und Schwächen ist entscheidend wichtig, um mögliche Probleme frühzeitig zu erkennen und anzugehen. Ein weiterer Vorteil liegt darin, die Benutzbarkeit und Anwenderfreundlichkeit weiter zu optimieren. Diese Fortschritte betonen die Bedeutung und die Möglichkeiten in diesem Bereich \cite{sedlmair2011information}.