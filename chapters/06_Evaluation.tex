\chapter{Evaluation}
\label{chap:evaluation}

\section{Einleitung}
Die Evaluation des entwickelten Systems ist entscheidend, um dessen Leistungsf\"ahigkeit, Benutzerfreundlichkeit und Sicherheit zu gew\"ahrleisten. Schwerpunktm\"a\ss ig wurden funktionale Tests durchgef\"uhrt, um die korrekte Verarbeitung und Visualisierung von Mitarbeitendengespr\"achs\-daten zu \"uberpr\"ufen, w\"ahrend gleichzeitig die Benutzererfahrung durch gezielte Tests der Interaktivit\"at und Benutzerfreundlichkeit intensiv untersucht wurde. Diese umfassende Analyse sch\"arft das Verst\"andnis f\"ur die St\"arken und Schw\"achen des Systems und leistet einen wertvollen Beitrag zur kontinuierlichen Verbesserung und Akzeptanz innerhalb der Organisation \cite{akinnuwesi2012framework, achter2016bigdata, angrave2016hr}.

\section{Testmethoden}
Die Evaluation wurde mithilfe verschiedener Testmethoden durchgef\"uhrt, um die Funktionalit\"at, Benutzerfreundlichkeit und Sicherheit des Systems zu pr\"ufen:
\begin{itemize}
    \item \textbf{Funktionstests:} \"Uberpr\"ufung, ob alle funktionalen Anforderungen erf\"ullt werden, z. B. schnelle Ladezeiten und korrekte Visualisierungen \cite{kirk2019datavisualisation}.
    \item \textbf{Usability-Tests:} Bewertung der Benutzerfreundlichkeit durch Fokusgruppen aus HR-Mitarbeitenden und F\"uhrungskr\"aften \cite{heikkila2018quantified}.
    \item \textbf{Sicherheitstests:} Validierung der Verschl\"usselung und Authentifizierungsmethoden, um den Schutz sensibler Daten sicherzustellen \cite{van2011employee}.
\end{itemize}

\section{Ergebnisse der Evaluation}
\subsection{Funktionstests}
Die Funktionalit\"at des Systems wurde anhand definierter Anwendungsf\"alle getestet, die reale Mitarbeitendengespr\"achs\-prozesse simulieren. Die Tests ergaben, dass:
\begin{itemize}
    \item Alle Kernanforderungen erf\"ullt wurden (z. B. schnelle Ladezeiten, korrekte Darstellung von Donut- und Radarcharts).
    \item Die Datenverarbeitung effizient und fehlerfrei erfolgte.
    \item Die Visualisierungskomponenten interaktiv und intuitiv nutzbar waren \cite{watson2014bigdata}.
\end{itemize}

\subsection{Usability-Tests}
Die Benutzerfreundlichkeit wurde durch Fokusgruppen evaluiert. Die Ergebnisse zeigten:
\begin{itemize}
    \item \textbf{Hohe Zufriedenheit:} Testpersonen lobten die intuitive Bedienbarkeit und klare Visualisierungen \cite{khairat2018impact}.
    \item \textbf{Interaktive Funktionen:} Tooltipps und Echtzeit-Feedback wurden als besonders hilfreich empfunden.
    \item \textbf{Responsives Design:} Das System war auf verschiedenen Ger\"aten (Desktop, Tablet, Smartphone) gleicherma\ss en gut nutzbar \cite{tarvainen2014kubios}.
\end{itemize}

\subsection{Vergleich mit bestehenden Tools}
Ein Vergleich mit bestehenden HR-Tools wie HRworks und Evalea verdeutlichte die St\"arken des entwickelten Systems:
\begin{table}[h!]
\centering
\caption{Vergleich des entwickelten Systems mit bestehenden Tools}
\label{tab:tool_comparison}
\begin{tabularx}{\textwidth}{|X|X|X|}
\hline
\textbf{Kriterium}              & \textbf{HRworks/Evalea}                                                                 & \textbf{Entwickeltes System}                                                          \\\hline
\textbf{Visualisierung}         & Eingeschr\"ankte Visualisierungsoptionen.                 & Umfangreiche interaktive Diagramme wie Donut- und Radarcharts.                      \\\hline
\textbf{Anpassbarkeit}          & Geringe Anpassungsm\"oglichkeiten.                       & Hohe Flexibilit\"at durch personalisierte Visualisierungen.                        \\\hline
\textbf{Datenverarbeitung}      & Standardisierte Workflows.                           & Optimierte Verarbeitung komplexer HR-Daten.                                         \\\hline
\textbf{Kosten}                 & H\"ohere Lizenzkosten.                                  & Kosteneffiziente Open-Source-Ans\"atze.                                             \\\hline
\end{tabularx}
\end{table}

\section{Kritische Reflexion}
\subsection{St\"arken des Systems}
\begin{itemize}
    \item Benutzerfreundlichkeit: Klare Benutzeroberfl\"ache und intuitive Navigation \cite{heer2012interactive}.
    \item Leistungsf\"ahigkeit: Schnelle Verarbeitung und Visualisierung gro\ss er Datenmengen \cite{chen2012interactive}.
    \item Sicherheit: Zuverl\"assige Authentifizierungs- und Verschl\"usselungsmechanismen \cite{boneder2023evaluation}.
\end{itemize}

\subsection{Schw\"achen des Systems}
\begin{itemize}
    \item Begrenzte Erweiterungen: Fehlende Integration von KI-gest\"utzten Analysetools \cite{tambe2019artificial}.
    \item Usability-Tests: Begrenzter Umfang der Tests, insbesondere in realen Anwendungsszenarien.
\end{itemize}

\subsection{Verbesserungspotenzial}
\begin{itemize}
    \item Erweiterung der Analysem\"oglichkeiten durch Machine-Learning-Modelle \cite{aral2012threeway}.
    \item Breitere Testabdeckung mit realen Unternehmensdaten.
    \item Optimierung der Benutzeroberfl\"ache basierend auf weiterem Nutzerfeedback \cite{sedlmair2011information}.
\end{itemize}

\section{Zusammenfassung}
Die Evaluation hat gezeigt, dass das entwickelte System eine solide Grundlage f\"ur die datengetriebene Analyse von Mitarbeitendengespr\"achs\-daten bietet. Die Kombination aus interaktiven Visualisierungen, effizienter Datenverarbeitung und hoher Benutzerfreundlichkeit unterstreicht den Mehrwert des Systems f\"ur HR-Abteilungen. Gleichzeitig zeigen die identifizierten Schw\"achen, dass Potenzial f\"ur Weiterentwicklungen besteht, insbesondere im Bereich der KI-Integration und der Praxisvalidierung. Die gewonnenen Erkenntnisse legen die Basis f\"ur zuk\"unftige Verbesserungen und Erweiterungen des Systems \cite{burnett2021future}.
