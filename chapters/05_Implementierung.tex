\chapter{Implementierung}
\label{chap:implementierung}

\section{Technologische Basis}
\begin{itemize}
    \item \textbf{Frontend:} Entwicklung mit React und TypeScript. Integration von Material-UI für UI-Komponenten.
    \item \textbf{Backend:} Aufbau von RESTful APIs mit .NET Core. Middleware für Authentifizierung und Fehlerbehandlung.
    \item \textbf{Datenbank:} Datenbankmodellierung und -abfragen in Azure SQL.
\end{itemize}

\section{Herausforderungen und Lösungen}
\begin{itemize}
    \item \textbf{Datenkonsistenz:} Einsatz von Transaktionen in der Datenbank zur Sicherstellung der Datenintegrität.
    \item \textbf{Echtzeit-Updates:} Nutzung von Azure Service Bus für asynchrone Nachrichtenverarbeitung.
    \item \textbf{Responsiveness:} Optimierung der Benutzeroberfläche für verschiedene Geräte.
\end{itemize}

\section{Visualisierungsimplementierung}
Integration von Chart.js für Donut- und Radarcharts.  
Beispiel: Erstellung eines Radardiagramms:
\begin{lstlisting}[language=JavaScript, caption=Erstellung eines Radardiagramms]
const data = {
  labels: ['Teamarbeit', 'Effizienz', 'Pünktlichkeit', 'Innovation'],
  datasets: [
    {
      label: 'Mitarbeiter A',
      data: [80, 90, 70, 85],
      backgroundColor: 'rgba(54, 162, 235, 0.2)',
      borderColor: 'rgba(54, 162, 235, 1)',
      borderWidth: 2,
    },
  ],
};
\end{lstlisting}

\section{Codebeispiele}
RESTful API-Endpunkt:
\begin{lstlisting}[language=C, caption=RESTful API-Endpunkt]
[HttpGet("api/employees/{id}")]
public IActionResult GetEmployeeData(int id)
{
    var employee = _context.Employees.Find(id);
    return Ok(employee);
}
\end{lstlisting}