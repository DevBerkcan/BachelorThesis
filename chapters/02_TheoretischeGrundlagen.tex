\chapter{Theoretische Grundlagen}
\label{chap:theoretische-grundlagen}

Die individuelle Entwicklung von Mitarbeitenden erfolgt maßgeblich durch zielgerichtete Gespräche und datengestützte Feedbackprozesse. Die folgenden Abschnitte beleuchten die Bedeutung von Mitarbeitendengesprächen, die Rolle der Datenvisualisierung in diesem Kontext sowie die technologische Basis für deren Implementierung. Hierbei wird auf die Verknüpfung von strategischen Unternehmenszielen und persönlichen Entwicklungsmöglichkeiten eingegangen, um eine nachhaltige Verbesserung der Teamdynamik und Leistungseffizienz zu fördern. Diese theoretischen Grundlagen bilden das Fundament für die anschließende Analyse und Entwicklung eines optimalen Systems zur Visualisierung von Mitarbeitendengesprächsdaten.

\section{Mitarbeitendengespräche und ihre Bedeutung}
Mitarbeitendengespräche (MAG) umfassen alle Gespräche, die Vorgesetzte mit Mitarbeitenden zu spezifischen Anlässen führen. Diese Gespräche haben mehrere zentrale Funktionen. Sie ermöglichen den Austausch wichtiger Informationen zwischen Mitarbeitenden und Führungskräften und bieten eine Gelegenheit, Wertschätzung auszudrücken, Lob auszusprechen oder konstruktive Kritik zu üben. Darüber hinaus dienen sie der gemeinsamen Formulierung von Entwicklungszielen, die sowohl die individuellen Bedürfnisse der Mitarbeitenden als auch die strategischen Anforderungen des Unternehmens berücksichtigen \cite{schober2008}.


Ein strukturierter Ablauf von Mitarbeitendengesprächen fördert die innerbetriebliche Kommunikation und stärkt die Bindung der Mitarbeitenden an das Unternehmen. Zu den zentralen Elementen solcher Gespräche gehört die Bewertung der bisherigen Zielerreichung, bei der der Fortschritt in Bezug auf zuvor gesetzte Ziele analysiert wird \cite{duarte2012performance}. Ebenso entscheidend ist das Setzen neuer Ziele, bei dem SMART-Ziele definiert werden, die spezifisch, messbar, akzeptiert, realistisch und terminiert sind, um eine klare Richtung für die zukünftige Arbeit vorzugeben \cite{duarte2012performance}. Darüber hinaus spielt die Diskussion von Entwicklungsmaßnahmen eine wichtige Rolle, bei der Schulungsbedarf und Weiterbildungsangebote identifiziert werden, um die persönliche und berufliche Entwicklung der Mitarbeitenden gezielt zu fördern \cite{bryson2011employee}. Eine strukturierte Gesprächsführung kann dabei Verzerrungen durch subjektive Einschätzungen minimieren, insbesondere durch den Einsatz datenbasierter Methoden \cite{heikkila2018}. Dies unterstützt Führungskräfte dabei, fundierte Entscheidungen zu treffen, die sowohl die individuellen Ziele der Mitarbeitenden als auch die strategischen Unternehmensziele berücksichtigen \cite{barton2012}.


\section{Relevanz von Datenvisualisierung}
Datenvisualisierung ist ein zentraler Bestandteil moderner Analytik und spielt eine entscheidende Rolle bei der Interpretation komplexer Informationen. Sie unterstützt Führungskräfte dabei, datenbasierte Entscheidungen zu treffen, indem sie große Datenmengen in verständliche visuelle Darstellungen umwandelt \cite{kirk2016data}. Eine der wichtigsten Stärken der Datenvisualisierung liegt in der verbesserten Entscheidungsfindung, da Daten schneller und effizienter interpretiert werden können \cite{kirk2016data}. Zudem ermöglicht sie die Erkennung von Trends und Abweichungen, indem historische und aktuelle Daten leicht verglichen werden können, um Entwicklungen zu identifizieren \cite{ware2012information}. Darüber hinaus fördern visuelle Darstellungen eine klare Kommunikation zwischen Teams und Stakeholdern, was die Verständlichkeit komplexer Inhalte erhöht und die Zusammenarbeit erleichtert \cite{evergreen2016effective}.

Im Kontext von Mitarbeitendengesprächen bieten sich insbesondere Radarcharts und Donutcharts an, da sie spezifische Vorteile für die Analyse und Darstellung bieten. Radarcharts ermöglichen den Vergleich mehrerer Leistungsdimensionen wie Teamarbeit, Effizienz und Pünktlichkeit und helfen so, Stärken und Schwächen auf einen Blick zu erkennen \cite{heikkila2018}. Donutcharts hingegen bieten eine übersichtliche Darstellung des Status von Zielvereinbarungen oder Fortschritten und erleichtern dadurch die Nachverfolgbarkeit und Analyse von Entwicklungen \cite{evergreen2016effective}.


\section{Technologische Basis}
Die technologische Basis für ein System zur Visualisierung von Mitarbeitendengesprächsdaten ist entscheidend für dessen Funktionalität, Benutzerfreundlichkeit und Skalierbarkeit. Die ausgewählten Technologien lassen sich in drei Hauptbereiche unterteilen: Frontend, Backend und unterstützende Dienste.

\subsection*{Frontend-Technologien}
Moderne Frontend-Technologien spielen eine zentrale Rolle bei der Entwicklung benutzerfreundlicher und reaktionsfähiger Benutzeroberflächen. Die gewählten Technologien für das System sind:

\begin{itemize}
    \item \textbf{React:}  
    React ist eine weit verbreitete JavaScript-Bibliothek, die durch ihre komponentenbasierte Architektur eine modulare und effiziente Entwicklung ermöglicht. Sie unterstützt die Erstellung wiederverwendbarer UI-Komponenten, wodurch die Entwicklungszeit verkürzt und die Wartbarkeit erhöht wird. Mit Hilfe von Virtual DOM bietet React eine hohe Performance, indem es nur die tatsächlich geänderten Teile des DOM aktualisiert \cite{stefanov2021react}. Studien zeigen, dass React insbesondere bei datenintensiven Anwendungen wie Dashboards oder interaktiven Visualisierungen effizient eingesetzt werden kann \cite{reactDocumentation}.

    \item \textbf{TypeScript:}  
    TypeScript erweitert JavaScript durch statische Typisierung und bietet Entwicklern die Möglichkeit, Fehler frühzeitig zu erkennen. Dies verbessert die Codequalität erheblich und reduziert die Anzahl der Laufzeitfehler. Zudem erleichtert TypeScript die Zusammenarbeit in Teams, da Typdefinitionen und Intellisense-Unterstützung in modernen IDEs wie Visual Studio Code die Entwicklung beschleunigen \cite{typeScriptDocumentation}. Untersuchungen zeigen, dass der Einsatz von TypeScript in Projekten die Wartbarkeit und Skalierbarkeit langfristig verbessert \cite{typescriptSurvey2021}.
\end{itemize}
\newpage

\subsection*{Backend-Technologien}
Die Auswahl moderner Backend-Technologien ist entscheidend für die Entwicklung einer skalierbaren, leistungsstarken und sicheren Serverarchitektur. Für das System wurden die folgenden Technologien gewählt:

\begin{itemize}
    \item \textbf{.NET Core:}  
    .NET Core ist ein plattformübergreifendes Framework, das sich durch hohe Performance und Flexibilität auszeichnet. Es bietet eine modulare Architektur und Unterstützung für asynchrone Programmierung, was die Entwicklung skalierbarer Backend-Anwendungen erleichtert. Mit der Unterstützung von Entity Framework Core ermöglicht .NET Core eine effiziente Datenbankintegration und Abfrageverarbeitung \cite{microsoftDotNet}. Studien zeigen, dass .NET Core aufgrund seiner plattformübergreifenden Kompatibilität und seiner effizienten Ressourcennutzung eine bevorzugte Wahl für Cloud-basierte Anwendungen ist \cite{dotNetPerformance2021}.

    \item \textbf{Microsoft Graph API:}  
    Die Microsoft Graph API bietet eine zentrale Schnittstelle für die Interaktion mit Microsoft-Diensten wie Azure Active Directory, Outlook und SharePoint. Sie ermöglicht den Zugriff auf Benutzer- und Organisationsdaten, was die Automatisierung von Prozessen und die Integration mit bestehenden Systemen erleichtert. Beispielsweise können Benutzerdaten für Mitarbeitendengespräche dynamisch abgerufen und mit Kalenderdaten verknüpft werden. Untersuchungen zeigen, dass der Einsatz der Graph API die Effizienz in datenintensiven Anwendungen erheblich steigern kann \cite{microsoftGraphAPI, smith2021graph}.
\end{itemize}


\subsection*{Azure-Dienste}
Microsoft Azure bietet eine Vielzahl von Diensten, die für die Implementierung einer skalierbaren und sicheren Lösung genutzt werden:
\begin{itemize}
    \item \textbf{Entra ID (ehemals Azure Active Directory):}  
    Entra ID bietet umfassende Authentifizierungs- und Autorisierungsdienste, einschließlich Multi-Faktor-Authentifizierung und rollenbasierter Zugriffskontrolle \cite{microsoftEntraID, woods2020authentication}.
    \item \textbf{MSSQL:}  
    Cloudbasierte Datenbanklösung, die Sicherheit und Skalierbarkeit für große Datenmengen bietet \cite{azureDocumentation}.
    \item \textbf{Application Insights:}  
    Ein Telemetriedienst zur Überwachung und Analyse der Anwendungsleistung \cite{microsoftAppInsights, li2021monitoring}.
    \item \textbf{App Services:}  
    Azure App Services bietet eine skalierbare und verwaltete Plattform für die Bereitstellung von Webanwendungen, APIs und mobilen Backends \cite{azureAppServices, thomas2021azure}.
    \item \textbf{Container Registry:}  
    Ein Dienst zur Verwaltung von Container-Images, der Automatisierung und Konsistenz zwischen Entwicklungs-, Test- und Produktionsumgebungen ermöglicht \cite{azureContainerRegistry, dockerContainers2020}.
\end{itemize}

\subsection*{Versionskontrolle und DevOps}
Die Integration von Versionskontroll- und DevOps-Tools stellt sicher, dass Entwicklungs- und Bereitstellungsprozesse effizient und nachvollziehbar sind:
\begin{itemize}
    \item \textbf{Git:}  
    Ein verteiltes Versionskontrollsystem, das die Zusammenarbeit erleichtert und die Nachverfolgbarkeit von Änderungen sicherstellt \cite{chacon2021git, loeliger2012git}.
    \item \textbf{DevOps:}  
    Eine Methodik zur Integration von Entwicklungs- und Betriebsteams, um die Softwarebereitstellung zu optimieren \cite{azureDevOps, kim2020devops}.
    \begin{itemize}
        \item \textbf{CI/CD-Pipeline:}  
        Automatisierte Continuous Integration und Continuous Deployment zur Optimierung von Entwicklungs- und Bereitstellungsprozessen \cite{fowler2020continuous}.
        \item \textbf{HBS-Masstransit Artefakt:}  
        Eine Erweiterung für Messaging-Lösungen in verteilten Systemen, basierend auf MassTransit und Azure Service Bus \cite{masstransit2021, masstransitAzure2022}.
        \item \textbf{Repository:}  
        Zentrale Plattform zur Versionierung und Verwaltung von Quellcode, Konfigurationsdateien und Build-Skripten \cite{azureRepos2023}.
    \end{itemize}
\end{itemize}

\subsection*{Integration der Technologien}
Die Wahl dieser Technologien wurde maßgeblich durch den bestehenden Technologie-Stack von Realcore beeinflusst. Dieser Stack basiert auf modernen, bewährten Tools, die bereits erfolgreich in anderen Projekten eingesetzt wurden \cite{chhajed2015elk, wilson2018mern}. Die vorgegebene Auswahl spiegelt die strategische Entscheidung wider, auf leistungsfähige und skalierbare Technologien zu setzen, die den Anforderungen moderner Softwareentwicklung gerecht werden. Durch diese Vorgaben wurde sichergestellt, dass das System optimal in die bestehende Infrastruktur integriert werden kann und gleichzeitig den neuesten technologischen Standards entspricht \cite{sharma2018fullstack}.

Die Kombination aus React und TypeScript ermöglicht die Entwicklung einer flexiblen und responsiven Benutzeroberfläche. Backend-seitig sorgt .NET Core für eine skalierbare und performante Architektur, während MS SQL als sichere Speicherlösung dient. Die enge Verzahnung dieser Technologien gewährleistet eine effiziente Datenverarbeitung und nahtlose Integration in bestehende Infrastrukturen. Die strategische Nutzung dieser modernen Technologien spiegelt Realcores Engagement für Innovation und Qualität wider \cite{mccreary2009xrx}.