\chapter{Theoretische Grundlagen}
\label{chap:theoretische-grundlagen}

\section{Mitarbeitendengespräche}
Mitarbeitendengespräche, auch „MAG“ genannt, umfassen alle Gespräche, die Vorgesetzte aus verschiedenen Anlässen mit ihren Mitarbeitenden führen. Ziel dieser Gespräche ist es, Informationen zu vermitteln, einen Meinungsaustausch zu ermöglichen sowie Wertschätzung, Lob und gegebenenfalls Kritik auszudrücken. Darüber hinaus bieten sie die Gelegenheit, Probleme zu analysieren und bevorstehende Veränderungen zu besprechen. Regelmäßige Mitarbeitendengespräche, die in der Regel ein- bis zweimal jährlich stattfinden, tragen wesentlich zur Verbesserung der innerbetrieblichen Kommunikation, Zusammenarbeit und Partizipation bei. Dabei ist die kommunikative Kompetenz der Führungskräfte eine wesentliche Grundlage für den Erfolg solcher Gespräche [1].

Typischerweise umfassen Mitarbeitendengespräche mehrere zentrale Elemente:
\begin{itemize}
    \item \textbf{Bewertung der bisherigen Zielerreichung:} Eine retrospektive Betrachtung, in der der Fortschritt in Bezug auf zuvor gesetzte Ziele analysiert wird.
    \item \textbf{Setzen neuer Ziele:} Definition von SMART-Zielen (spezifisch, messbar, akzeptiert, realistisch, terminiert), die die Grundlage für die zukünftige Arbeit bilden.
    \item \textbf{Diskussion von Entwicklungsmaßnahmen:} Identifikation von Schulungsbedarf, Weiterbildungsangeboten und anderen Maßnahmen, die das persönliche und berufliche Wachstum fördern.
\end{itemize}

Eine strukturierte Dokumentation dieser Gespräche ist unerlässlich, um die Entwicklung der Mitarbeitenden langfristig nachzuvollziehen und strategische Entscheidungen auf einer soliden Datenbasis zu treffen. Die Dokumentation ermöglicht zudem eine einheitliche Bewertung und Vergleichbarkeit von Leistungen, was besonders in größeren Organisationen von Bedeutung ist. Ohne eine effektive Datenaufbereitung bleibt jedoch das Potenzial vieler Gespräche ungenutzt.

\section{Relevanz von Datenvisualisierung}
Datenvisualisierung ist ein wesentlicher Bestandteil moderner Analytik und spielt eine entscheidende Rolle bei der Interpretation komplexer Informationen. Sie ermöglicht es, große Datenmengen in visuelle Formate umzuwandeln, die einfacher zu verstehen und zu analysieren sind. Im Kontext von Mitarbeitendengesprächen kann die Visualisierung dazu beitragen, Muster und Zusammenhänge aufzudecken, die durch reine Text- oder Tabellenform schwer erkennbar sind.

Einige der Hauptvorteile von Datenvisualisierung umfassen:
\begin{itemize}
    \item \textbf{Verbesserte Entscheidungsfindung:} Führungskräfte können fundierte Entscheidungen treffen, da sie die Daten auf einen Blick verstehen.
    \item \textbf{Erkennung von Trends und Abweichungen:} Historische und aktuelle Daten können leicht verglichen werden, um positive oder negative Entwicklungen zu identifizieren.
    \item \textbf{Kommunikation und Präsentation:} Datenvisualisierung erleichtert die Präsentation von Ergebnissen und fördert eine klare Kommunikation zwischen Teams.
\end{itemize}

Insbesondere Donut- und Radarcharts sind für die Visualisierung von Mitarbeitendengesprächsdaten ideal geeignet:
\begin{itemize}
    \item \textbf{Donut-Charts:} Sie bieten einen schnellen Überblick über den Status von Zielvereinbarungen, z. B. den Anteil erfüllter, in Bearbeitung befindlicher oder nicht erreichter Ziele.
    \item \textbf{Radarcharts:} Sie ermöglichen einen direkten Vergleich mehrerer Leistungsdimensionen, wie Teamarbeit, Effizienz, Pünktlichkeit und Innovation. Diese Diagramme helfen, Stärken und Schwächen aufzuzeigen und Prioritäten für Verbesserungsmaßnahmen zu setzen.
\end{itemize}

Die visuelle Darstellung macht es für Führungskräfte einfacher, komplexe Datensätze zu interpretieren und konkrete Maßnahmen abzuleiten.

\section{Technologien und Frameworks}
Für die Entwicklung des Tools wurden moderne Technologien und Frameworks ausgewählt, die den Anforderungen an Skalierbarkeit, Benutzerfreundlichkeit und Flexibilität gerecht werden. Im Folgenden werden die wichtigsten Technologien vorgestellt:

\begin{itemize}
    \item \textbf{React:} Ein beliebtes JavaScript-Framework für die Entwicklung von Benutzeroberflächen. React ermöglicht eine komponentenbasierte Entwicklung, die Wiederverwendbarkeit und Wartbarkeit fördert. Durch die Virtual-DOM-Technologie werden Änderungen effizient verarbeitet, was die Performance steigert.
    \item \textbf{TypeScript:} Eine typsichere Programmiersprache, die auf JavaScript aufbaut. TypeScript reduziert Fehler im Code durch statische Typisierung und erleichtert die Wartung großer Codebasen. Es verbessert die Lesbarkeit und macht den Entwicklungsprozess robuster.
    \item \textbf{.NET Core:} Ein plattformunabhängiges Framework zur Entwicklung moderner Webanwendungen. .NET Core bietet eine hohe Performance und ermöglicht die Erstellung von RESTful APIs, die eine einfache Integration zwischen Frontend und Backend gewährleisten.
    \item \textbf{Azure SQL:} Eine cloudbasierte Datenbanklösung, die hohe Verfügbarkeit und Skalierbarkeit bietet. Azure SQL ermöglicht eine schnelle Abfrage großer Datenmengen und stellt Sicherheitsfunktionen wie Verschlüsselung und Zugriffskontrolle bereit.
\end{itemize}

Die Wahl dieser Technologien basiert auf einer detaillierten Analyse der Projektanforderungen:
\begin{itemize}
    \item \textbf{Flexibilität:} React und TypeScript bieten die nötige Flexibilität, um ein dynamisches und responsives Frontend zu entwickeln.
    \item \textbf{Skalierbarkeit:} Mit .NET Core und Azure SQL kann das System auf wachsende Daten- und Nutzerzahlen skaliert werden.
    \item \textbf{Benutzerfreundlichkeit:} Die Kombination dieser Technologien ermöglicht eine intuitive Benutzeroberfläche und eine effiziente Verarbeitung der Daten.
\end{itemize}

Zusammen bieten diese Technologien eine solide Grundlage für die Entwicklung eines robusten und benutzerfreundlichen Reporting-Tools, das den Anforderungen von Personalabteilungen und Führungskräften gerecht wird.
