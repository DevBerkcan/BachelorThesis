\chapter{Einleitung}
\label{chap:einleitung}

\section{Motivation}
Die Qualität und Effizienz von Mitarbeitendengesprächen spielen eine entscheidende Rolle für den Erfolg eines Unternehmens. Sie bieten nicht nur eine Plattform für die Rückmeldung zur individuellen Leistung, sondern auch für die Festlegung zukünftiger Ziele und die Entwicklungsperspektiven der Mitarbeitenden. In einer zunehmend datengetriebenen Unternehmenswelt wird es immer wichtiger, die Ergebnisse solcher Gespräche effektiv zu erfassen, auszuwerten und nutzbar zu machen.

Trotz dieser Bedeutung bleibt die Analyse und Nachverfolgung der Ergebnisse häufig unstrukturiert. In vielen Fällen werden Gesprächsdaten manuell dokumentiert, beispielsweise in Tabellen oder Textdokumenten. Diese Vorgehensweise ist jedoch nicht nur zeitaufwändig, sondern auch anfällig für menschliche Fehler. Ohne geeignete Tools zur Analyse und Visualisierung werden die Daten oft nicht optimal genutzt, wodurch wertvolle Erkenntnisse verloren gehen.

Durch den Einsatz moderner Technologien zur Datenvisualisierung können jedoch Muster und Trends aus den Gesprächsdaten abgeleitet werden, die neue Einblicke und eine datenbasierte Entscheidungsfindung ermöglichen. Beispielsweise können Führungskräfte durch interaktive Dashboards die Entwicklung ihrer Teams über mehrere Jahre hinweg nachvollziehen und Bereiche identifizieren, die Verbesserungspotenzial aufweisen. Diese Arbeit widmet sich der Entwicklung eines solchen Tools, das den gestiegenen Anforderungen an die Datenanalyse gerecht wird und gleichzeitig benutzerfreundlich ist.

\section{Problemstellung}
In vielen Unternehmen werden Mitarbeitendengesprächsdaten zwar erfasst, aber nicht effektiv analysiert oder visualisiert. Ein Großteil der Informationen verbleibt in statischen Berichten oder Tabellen, was eine datengetriebene Nutzung der Ergebnisse erschwert. Die fragmentierte Speicherung und mangelnde Integration in bestehende HR-Prozesse führen dazu, dass wichtige Erkenntnisse oft übersehen oder gar nicht erst generiert werden.

Besonders Führungskräfte und Personalabteilungen stehen vor der Herausforderung, aus einer Vielzahl von Daten sinnvolle Schlussfolgerungen zu ziehen. Intuitive Visualisierungen können helfen, komplexe Daten verständlich darzustellen, doch fehlen in vielen Organisationen die Werkzeuge, die diesen Bedarf abdecken. Hinzu kommt, dass die manuelle Aufbereitung von Daten zeitaufwändig und fehleranfällig ist. 

Es fehlt an einer Lösung, die Gesprächsdaten nicht nur speichert, sondern auch interaktiv und visuell ansprechend aufbereitet, um die Analyse zu vereinfachen und fundierte Entscheidungen zu fördern. Ein solches Tool könnte nicht nur die Effizienz in HR-Prozessen steigern, sondern auch die Qualität von Mitarbeitendengesprächen nachhaltig verbessern.

\section{Zielsetzung}
Das Ziel dieser Arbeit ist die Entwicklung eines Reporting-Tools, das Mitarbeitendengesprächsdaten strukturiert erfasst, speichert und visuell darstellt. Dabei liegt der Fokus auf der Benutzerfreundlichkeit und der Möglichkeit, die Daten interaktiv zu analysieren. Führungskräfte sollen in der Lage sein, Gespräche über mehrere Zeiträume hinweg zu vergleichen, Trends zu erkennen und die Leistung ihrer Teams besser zu bewerten.

Das Tool wird speziell für die Anforderungen von Personalabteilungen und Führungskräften entwickelt. Es ermöglicht die Darstellung der Daten in Form von Donut-Charts, die einen schnellen Überblick über den Status von Zielvereinbarungen geben, und Radarcharts, die eine detaillierte Analyse einzelner Leistungsdimensionen erlauben. Darüber hinaus soll das System skalierbar sein, um mit wachsenden Datenmengen und neuen Anforderungen der Nutzer Schritt halten zu können. 

\section{Aufbau der Arbeit}
Die vorliegende Arbeit ist wie folgt aufgebaut:
\begin{itemize}
    \item \textbf{Kapitel 2:} Behandelt die theoretischen Grundlagen und erläutert die Bedeutung von Mitarbeitendengesprächen sowie die Relevanz moderner Datenvisualisierung. Darüber hinaus wird auf die verwendeten Technologien eingegangen, die bei der Entwicklung des Tools eingesetzt werden.
    \item \textbf{Kapitel 3:} Analysiert die funktionalen und nicht-funktionalen Anforderungen an das System. Zudem werden die Ergebnisse aus Stakeholder-Interviews und Umfragen vorgestellt, die zur Bedarfsanalyse durchgeführt wurden.
    \item \textbf{Kapitel 4:} Beschreibt die Konzeption des Systems, einschließlich der Systemarchitektur, des Datenmodells und des Visualisierungsdesigns. Auch Sicherheitskonzepte werden in diesem Kapitel behandelt.
    \item \textbf{Kapitel 5:} Widmet sich der technischen Implementierung des Tools und stellt die verwendeten Technologien, die Implementierungsdetails und die Herausforderungen bei der Entwicklung vor.
    \item \textbf{Kapitel 6:} Evaluiert das entwickelte System anhand von Funktionalitäts- und Usability-Tests. Dabei werden sowohl die Ergebnisse als auch mögliche Verbesserungspotenziale analysiert.
    \item \textbf{Kapitel 7:} Fasst die Ergebnisse der Arbeit zusammen, diskutiert potenzielle Weiterentwicklungen und gibt einen Ausblick auf die zukünftige Anwendung des Tools in der Praxis.
\end{itemize}
