\chapter{Theoretische Grundlagen}
\label{chap:theoretische-grundlagen}

Die individuelle Entwicklung von Mitarbeitenden erfolgt maßgeblich durch zielgerichtete Gespräche und datengestützte Feedbackprozesse. Diese Ansätze fördern nicht nur die persönliche Entwicklung, sondern tragen auch zur Erreichung strategischer Unternehmensziele bei. Zielgerichtete Gespräche, gepaart mit Feedback, bieten eine effektive Grundlage, um individuelle Entwicklungsmöglichkeiten zu identifizieren und die Teamdynamik sowie die Leistungseffizienz nachhaltig zu verbessern \cite{locke2002goal}. Die folgenden Abschnitte beleuchten die Bedeutung von Mitarbeitendengesprächen, die Rolle der Datenvisualisierung in diesem Kontext sowie die technologische Basis für deren Implementierung. Diese theoretischen Grundlagen bilden das Fundament für die anschließende Analyse und Entwicklung eines optimalen Systems zur Visualisierung von Mitarbeitendengesprächsdaten.


\section{Mitarbeitendengespräche: Struktur, Ziele und Nutzen}
Mitarbeitendengespräche (\hyperref[abkuerzungen]{MAG}) umfassen alle Gespräche, die Vorgesetzte mit Mitarbeitenden zu spezifischen Anlässen führen. Diese Gespräche haben mehrere zentrale Funktionen. Sie ermöglichen den Austausch wichtiger Informationen zwischen Mitarbeitenden und Führungskräften und bieten eine Gelegenheit, Wertschätzung auszudrücken, Lob auszusprechen oder konstruktive Kritik zu üben. Darüber hinaus dienen sie der gemeinsamen Formulierung von Entwicklungszielen, die sowohl die individuellen Bedürfnisse der Mitarbeitenden als auch die strategischen Anforderungen des Unternehmens berücksichtigen.

Laut Jochum \cite{jochum2019} sind Mitarbeitendengespräche ein wesentliches Führungsinstrument, um Mitarbeitende gezielt zu motivieren und deren Leistungsfähigkeit nachhaltig zu steigern. Eine strukturierte Gesprächsführung verbessert nicht nur die Kommunikation, sondern stärkt auch das Vertrauen zwischen Führungskräften und Mitarbeitenden \cite{unigraz2020}. Darüber hinaus zeigt die Analyse von Wikipedia \cite{wikiMAG2023}, dass Mitarbeitendengespräche einen wesentlichen Beitrag zur strategischen Personalentwicklung leisten und langfristig zur Unternehmensentwicklung beitragen.


Ein gut strukturierter Ablauf von (\hyperref[abkuerzungen]{MAG}) fördert die interne Kommunikation und stärkt die Bindung der Mitarbeiter an das Unternehmen. Zu den Schlüsselelementen solcher Gespräche zählt die Bewertung des bisherigen Zielerreichungsstandards mit einer Analyse des Fortschritts im Hinblick auf vorher festgelegte Ziele.  \cite{duarte2012performance}. Ebenso wichtig ist das Festlegen neuer Ziele unter Verwendung der SMART-Ziele - also Ziele mit klaren Vorgaben hinsichtlich Spezifität (specific), Messbarkeit (measurable), Akzeptanz (achievable), Realismus (realistic) und Terminierung (timely) - um eine klare Ausrichtung für kommende Arbeitsschritte zu schaffen.  \cite{duarte2012performance}. Zusätzlich ist es wichtig zu betonen die Bedeutung von Entwicklungsmaßnahmen im Gespräch um den Schulungsbedarf und Weiterbildungsangebote zu ermitteln und so die persönliche und berufliche Entwicklung der Mitarbeiter gezielt zu fördern. \cite{bryson2011employee}. Durch eine strukturierte Gesprächsführung können subjektive Einschätzungen minimiert werden insbesondere durch den Einsatz von datengestützten Methodiken. \cite{heikkila2018}. Dies hilft Führungskräften dabei informierte Entscheidungen zu treffen die sowohl die individuellen Ziele der Mitarbeiter als auch die strategischen Unternehmensziele berücksichtigen. \cite{barton2012}.


\section{Relevanz von Datenvisualisierung}
Datenvisualisierung bildet die Brücke zwischen rohen Daten und deren praktischer Nutzung im Entscheidungsprozess. Sie ermöglicht Führungskräften und Mitarbeitenden, komplexe Datenmuster auf intuitive Weise zu verstehen und darauf basierend fundierte Entscheidungen zu treffen. Eine aktuelle Studie zeigt, dass Führungskräfte visuelle Analysen herkömmlichen Tabellen vorziehen, da visuelle Darstellungen die Informationsverarbeitung um bis zu 60\% beschleunigen können \cite{saket2017task}.

Eine der zentralen Stärken der Datenvisualisierung ist ihre Fähigkeit, Trends, Muster und Anomalien in großen Datenmengen schnell zu identifizieren \cite{kirk2016data}. Dies ist insbesondere im HR-Bereich von Bedeutung, wo Mitarbeitendengesprächsdaten oft in unstrukturierter Form vorliegen. Hier bieten Radarcharts und Donutcharts spezifische Vorteile: Radarcharts visualisieren Leistungsdimensionen wie Teamarbeit, Effizienz und Pünktlichkeit auf einer radialsymmetrischen Achse, was die Identifikation von Schwächen und Stärken erleichtert \cite{heikkila2018}. Im Gegensatz dazu bieten Donutcharts eine kompakte Übersicht über Fortschritte und Zielerreichungen, die durch interaktive Elemente weiter angereichert werden können \cite{evergreen2016effective}.

Obwohl Radarcharts und Donutcharts in vielen Anwendungsfällen effektiv sind, gibt es Einschränkungen. Beispielsweise können Radarcharts bei zu vielen Dimensionen überladen wirken, was die Lesbarkeit einschränkt. Alternativ könnten Heatmaps oder parallele Koordinatenplots eingesetzt werden, um größere Datenmengen besser zu visualisieren \cite{ware2012information}. Diese ergänzenden Ansätze könnten in zukünftigen Weiterentwicklungen des Systems berücksichtigt werden.

\section{Technologische Basis}
Die technologische Basis für ein System zur Visualisierung von Mitarbeitendengesprächsdaten ist entscheidend für dessen Funktionalität, Benutzerfreundlichkeit und Skalierbarkeit. Eine durchdachte Technologieauswahl ermöglicht es, die Verarbeitung großer Datenmengen effizient zu gestalten, benutzerfreundliche Schnittstellen zu bieten und eine flexible Anpassung an wachsende Anforderungen sicherzustellen. Die eingesetzten Technologien lassen sich in drei Hauptbereiche unterteilen: Frontend, Backend und unterstützende Dienste \cite{fowler2002patterns}.



\subsection*{Frontend-Technologien}
Moderne Frontend-Technologien spielen eine zentrale Rolle bei der Entwicklung benutzerfreundlicher und reaktionsfähiger Benutzeroberflächen. Die gewählten Technologien für das System sind:
\begin{itemize}
    \item \textbf{React:}  
    React ist eine weit verbreitete JavaScript-Bibliothek, die durch ihre komponentenbasierte Architektur eine modulare und effiziente Entwicklung ermöglicht. Sie unterstützt die Erstellung wiederverwendbarer (\hyperref[abkuerzungen]{UI})-Komponenten, wodurch die Entwicklungszeit verkürzt und die Wartbarkeit erhöht wird. Mit Hilfe von Virtual (\hyperref[abkuerzungen]{DOM}) bietet React eine hohe Performance, indem es nur die tatsächlich geänderten Teile des DOM aktualisiert \cite{stefanov2021react}.


  \item \textbf{TypeScript:}  
TypeScript erweitert JavaScript durch statische Typisierung und bietet Entwicklern die Möglichkeit, Fehler frühzeitig zu erkennen. Dies verbessert die Codequalität erheblich und reduziert die Anzahl der Laufzeitfehler. Zudem erleichtert TypeScript die Zusammenarbeit in Teams, da Typdefinitionen und Intellisense-Unterstützung in modernen (\hyperref[abkuerzungen]{IDEs}) wie Visual Studio Code die Entwicklung beschleunigen \cite{typescriptDocumentation}. Untersuchungen zeigen, dass der Einsatz von TypeScript in Projekten die Wartbarkeit und Skalierbarkeit langfristig verbessert \cite{typescriptSurvey2021}.

\end{itemize}
\newpage

\subsection*{Backend-Technologien}
Die Auswahl moderner Backend-Technologien spielt eine entscheidende Rolle bei der Entwicklung des Systems. Eine zentrale Technologie ist .NET Core, das aufgrund seiner hohen Leistung und Anpassungsfähigkeit als plattformübergreifendes Framework ausgewählt wurde. Mit seiner modularen Struktur und der Unterstützung für asynchrone Programmierung erleichtert es die Erstellung skalierbarer Backend-Anwendungen. Darüber hinaus ermöglicht die Integration von Entity Framework Core (\hyperref[abkuerzungen]{EF Core}) eine effiziente Einbindung von Datenbankfunktionalitäten sowie die Verarbeitung von Abfragen \cite{microsoftDotNet}. Studien zeigen, dass .NET Core durch seine plattformübergreifende Kompatibilität und effiziente Ressourcennutzung eine bevorzugte Wahl für cloudbasierte Anwendungen ist \cite{dotNetPerformance2021}.

\subsection*{Azure-Dienste}
Microsoft Azure bietet eine Vielzahl von Diensten, die für die Implementierung einer skalierbaren und sicheren Lösung genutzt werden. Einer der zentralen Dienste ist Azure Active Directory, der umfassende Authentifizierungs- und Autorisierungsfunktionen, einschließlich Multi-Faktor-Authentifizierung (MFA), bereitstellt und somit die Sicherheit von Benutzerzugriffen gewährleistet \cite{microsoftEntraID}. Für die Datenverwaltung wird Azure MSSQL eingesetzt, eine cloudbasierte Datenbanklösung, die sich durch hohe Sicherheit und Skalierbarkeit auszeichnet und große Datenmengen effizient verarbeitet \cite{azureDocumentation}. Zur Überwachung und Analyse der Anwendungsleistung kommt Azure Application Insights zum Einsatz. Dieser Telemetriedienst ermöglicht es, die Performance von Anwendungen zu analysieren und potenzielle Probleme frühzeitig zu erkennen \cite{microsoftAppInsights, li2021monitoring}. Darüber hinaus stellt Azure App Services eine skalierbare und verwaltete Plattform für die Bereitstellung von Webanwendungen, APIs und mobilen Backends bereit, was die Entwicklung und Verwaltung moderner Anwendungen erheblich erleichtert \cite{azureAppServices, thomas2021azure}. Ergänzend dazu wird die Azure Container Registry verwendet, um Container-Images effizient zu verwalten. Dieser Dienst gewährleistet Automatisierung und Konsistenz zwischen Entwicklungs-, Test- und Produktionsumgebungen, was insbesondere für containerisierte Anwendungen von Vorteil ist \cite{azureContainerRegistry, dockerContainers2020}.

\subsection*{Versionskontrolle und DevOps}
Die Integration von Versionskontroll- und DevOps-Tools stellt sicher, dass Entwicklungs- und Bereitstellungsprozesse effizient und nachvollziehbar gestaltet werden. Git dient dabei als zentrales, verteiltes Versionskontrollsystem, das die Zusammenarbeit im Team erleichtert und die Nachverfolgbarkeit von Änderungen sicherstellt \cite{chacon2021git, loeliger2012git}. Zusätzlich ermöglicht die Anwendung von DevOps-Methodiken eine nahtlose Integration von Entwicklungs- und Betriebsteams, was die Softwarebereitstellung optimiert \cite{azureDevOps, kim2020devops}. Ein wichtiger Bestandteil dieser Methodik ist die Implementierung von CI/CD-Pipelines. Diese automatisierten Prozesse für Continuous Integration (CI) und Continuous Deployment (CD) ermöglichen es, Entwicklungs- und Bereitstellungszyklen zu beschleunigen und gleichzeitig die Qualität des Codes sicherzustellen \cite{fowler2020continuous}. 

Im Rahmen des Systems wurde außerdem ein HBS-Masstransit-Artefakt entwickelt, das Messaging-Lösungen in verteilten Systemen mithilfe von MassTransit und Azure Service Bus erweitert \cite{masstransit2021, masstransitAzure2022}. Schließlich spielt ein zentrales Repository eine entscheidende Rolle für die Versionierung und Verwaltung von Quellcode, Konfigurationsdateien und Build-Skripten. Dieses Repository ermöglicht eine effiziente Organisation und stellt sicher, dass alle Projektressourcen zentral verfügbar sind \cite{azureRepos2023}.


\subsection*{Integration der Technologien} Die Wahl dieser Technologien wurde maßgeblich durch den bestehenden Technologie-Stack von Realcore beeinflusst. Dieser Stack basiert auf modernen, bewährten Tools, die bereits erfolgreich in anderen Projekten eingesetzt wurden \cite{chhajed2015elk, wilson2018mern}. Die vorgegebene Auswahl spiegelt die strategische Entscheidung wider, auf leistungsfähige und skalierbare Technologien zu setzen, die den Anforderungen moderner Softwareentwicklung gerecht werden. Durch diese Vorgaben wurde sichergestellt, dass das System optimal in die bestehende Infrastruktur integriert werden kann und gleichzeitig den neuesten technologischen Standards entspricht \cite{sharma2018fullstack}.

Die Kombination aus React und TypeScript ermöglicht die Entwicklung einer flexiblen und responsiven Benutzeroberfläche. Backend-seitig sorgt .NET Core für eine skalierbare und performante Architektur, während MS (\hyperref[abkuerzungen]{SQL}) als sichere Speicherlösung dient. Die enge Verzahnung dieser Technologien gewährleistet eine effiziente Datenverarbeitung und nahtlose Integration in bestehende Infrastrukturen. Die strategische Nutzung dieser modernen Technologien spiegelt Realcores Engagement für Innovation und Qualität wider \cite{mccreary2009xrx}.