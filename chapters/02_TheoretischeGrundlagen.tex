\chapter{Theoretische Grundlagen}
\label{chap:theoretische-grundlagen}

\section{Mitarbeitendengespräche}
Mitarbeitendengespräche sind ein zentrales Instrument der Personalführung und dienen sowohl der Rückmeldung zur individuellen Leistung als auch der Entwicklung von Mitarbeitenden. Typischerweise umfassen diese Gespräche die Bewertung der bisherigen Zielerreichung, das Setzen neuer Ziele und die Diskussion von Entwicklungsmaßnahmen.

Eine strukturierte Dokumentation dieser Gespräche ist unerlässlich, um langfristig die Entwicklung der Mitarbeitenden und die Effizienz des Gesprächsprozesses zu beurteilen.

\section{Relevanz von Datenvisualisierung}
Datenvisualisierung ist ein wesentlicher Bestandteil moderner Analytik, da sie komplexe Informationen in einer intuitiv verständlichen Form präsentiert. Während Tabellen und Berichte häufig schwer zu interpretieren sind, können Diagramme und andere visuelle Elemente Muster und Zusammenhänge deutlich schneller aufzeigen. 

Insbesondere Donut- und Radarcharts eignen sich hervorragend, um Leistung und Zielerreichung auf einen Blick zu verdeutlichen. Diese Diagrammtypen ermöglichen es, sowohl aggregierte Daten als auch detaillierte Leistungsbewertungen übersichtlich darzustellen.

\section{Technologien und Frameworks}
Für die Entwicklung des Tools kommen folgende Technologien und Frameworks zum Einsatz:
\begin{itemize}
    \item \textbf{React:} Ein leistungsfähiges Frontend-Framework für die Entwicklung von interaktiven Benutzeroberflächen.
    \item \textbf{TypeScript:} Eine typsichere Programmiersprache, die Fehler im Code reduziert und die Wartbarkeit verbessert.
    \item \textbf{.NET Core:} Ein plattformunabhängiges Framework für die Entwicklung skalierbarer Backend-Anwendungen.
    \item \textbf{Azure SQL:} Eine cloudbasierte Datenbanklösung, die hohe Performance und Verfügbarkeit gewährleistet.
\end{itemize}

Die Wahl dieser Technologien basiert auf ihrer Flexibilität, Skalierbarkeit und ihrer Eignung für die Realisierung moderner Webanwendungen. Sie ermöglichen es, eine robuste und benutzerfreundliche Softwarelösung zu entwickeln, die den Anforderungen der Zielgruppe gerecht wird.
