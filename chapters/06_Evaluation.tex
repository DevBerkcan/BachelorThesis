\chapter{Evaluation}
\label{chap:evaluation}

Die Evaluation des entwickelten Systems ist entscheidend, um dessen Leistungsfähigkeit, Benutzerfreundlichkeit und Sicherheit zu gewährleisten. Schwerpunktmäßig wurden funktionale Tests durchgeführt, um die korrekte Verarbeitung und Visualisierung von Mitarbeitendengesprächsdaten zu überprüfen, während gleichzeitig die Benutzererfahrung durch gezielte Tests der Interaktivität und Benutzerfreundlichkeit intensiv untersucht wurde. Diese umfassende Analyse schärft das Verständnis für die Stärken und Schwächen des Systems und leistet einen wertvollen Beitrag zur kontinuierlichen Verbesserung und Akzeptanz innerhalb der Organisation \cite{akinnuwesi2012framework, achter2016bigdata, angrave2016hr}.

Die Evaluation wurde mithilfe verschiedener Testmethoden durchgeführt, um die Funktionalität, Benutzerfreundlichkeit und Sicherheit des Systems zu prüfen. Funktionstests überprüften, ob alle funktionalen Anforderungen erfüllt wurden, beispielsweise schnelle Ladezeiten und korrekte Visualisierungen \cite{kirk2019datavisualisation}. Usability-Tests, die mit Fokusgruppen aus HR-Mitarbeitenden und Führungskräften durchgeführt wurden, bewerteten die Benutzerfreundlichkeit \cite{heikkila2018quantified}. Sicherheitstests validierten die Verschlüsselung und Authentifizierungsmethoden, um den Schutz sensibler Daten sicherzustellen \cite{van2011employee}.

Die Ergebnisse zeigten, dass alle Kernanforderungen erfüllt wurden, darunter schnelle Ladezeiten, eine fehlerfreie Datenverarbeitung und interaktive, intuitiv nutzbare Visualisierungskomponenten \cite{watson2014bigdata}. Tests mit 40 Mitarbeitenden der RealCore Group belegten eine hohe Zufriedenheit der Nutzenden mit der intuitiven Bedienbarkeit und klaren Struktur der Visualisierungen. Die Testpersonen identifizierten jedoch auch Verbesserungspotenziale, wie erweiterte Filteroptionen und personalisierbare Dashboards. Im Vergleich mit bestehenden HR-Tools wie HRworks und Evalea zeigte das entwickelte System deutliche Vorteile, insbesondere bei der Flexibilität der Visualisierungen, der Verarbeitung komplexer Daten und den geringeren Kosten durch den Einsatz von Open-Source-Technologien.

Die Stärken des Systems liegen in der klaren Benutzeroberfläche, der schnellen Verarbeitung großer Datenmengen und den zuverlässigen Sicherheitsmechanismen \cite{heer2012interactive, chen2012interactive, boneder2023evaluation}. Schwächen wurden jedoch ebenfalls identifiziert, insbesondere die fehlende Integration von KI-gestützten Analysetools, die breitere Analysemöglichkeiten bieten könnten \cite{tambe2019artificial}. Die begrenzte Reichweite der Usability-Tests erschwert zudem eine umfassende Beurteilung der Praxistauglichkeit in realen Anwendungsszenarien.

Zukünftige Entwicklungen sollten auf der Integration von Machine-Learning-Modellen basieren, um die Analysefähigkeit zu erweitern und prädiktive Analysen zu ermöglichen \cite{aral2012threeway}. Breitere Testabdeckungen mit realen Unternehmensdaten würden die Validität und Zuverlässigkeit des Systems weiter steigern. Eine kontinuierliche Optimierung der Benutzeroberfläche durch Nutzerfeedback könnte die Bedienbarkeit weiter verbessern und zur Erhöhung der Nutzerakzeptanz beitragen \cite{sedlmair2011information}.

Die Evaluation hat gezeigt, dass das entwickelte System eine solide Grundlage für die datengetriebene Analyse von Mitarbeitendengesprächsdaten bietet. Die Kombination aus interaktiven Visualisierungen, effizienter Datenverarbeitung und hoher Benutzerfreundlichkeit unterstreicht den Mehrwert des Systems für HR-Abteilungen. Gleichzeitig legen die identifizierten Schwächen die Basis für zukünftige Verbesserungen und Erweiterungen des Systems \cite{burnett2021future}.
